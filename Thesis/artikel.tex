\documentclass[twocolumn]{phdsymp} %!PN

\usepackage{times}
\usepackage{graphicx} %pakket voor includeren figuren
\usepackage{epsfig}
\usepackage{amsmath}
\usepackage{amssymb}
\usepackage{url}


\def\BibTeX{{\rm B\kern-.05em{\sc i\kern-.025em b}\kern-.08em
    T\kern-.1667em\lower.7ex\hbox{E}\kern-.125emX}}


\begin{document}

\title{Similarity-based Ordering of\\Images in Search Engines} %!PN

\author{Klaas Bosteels}
% \thanks{K.~Bosteels is a Computer Science student at
%     Ghent University (UGent), Gent,
% Belgium. E-mail: Klaas.Bosteels@UGent.be .}}

\supervisor{Etienne Kerre, Val\'erie De Witte, Stefan Schulte}

\maketitle

\begin{abstract}
Search engines such as Google and Yahoo allow internet users to search through huge collections of 
images. These existing search engines are all text-based, i.e.~the searching is based on 
the comparison of keywords with textual annotations. Content-based image retrieval (CBIR) 
is a more natural approach in which the searching is based on the actual content of the images. 
However, CBIR systems are still not capable of searching through collections
consisting of millions of images. Therefore they are not suitable yet as a basis for a search engine
on the internet. As a solution to that problem, we introduce a hybrid approach that combines
text-based image retrieval (TBIR) with CBIR, without sacrificing the 
scalability of the system. More precisely we propose an extension of TBIR that adds the 
a\-bi\-li\-ty to sort the search results according to their similarity with a given set of 
examples. Such an extension requires measures that are able to express the similarity
between two images. Because similarity is a gradual concept, we use fuzzy set theory to 
model these measures. 
\end{abstract}

\begin{keywords}
similarity measures, colour images, fuzzy sets, content-based image retrieval
\end{keywords}

\section{Introduction}
\PARstart{M}{ost} CBIR systems use the query-by-example paradigm, i.e.~the user
supplies an example image and the system searches for images that are similar to the example.
The approach introduced in this paper adds comparable functionality to a traditional 
TBIR system. A query is still initiated by providing one or more keywords, but as soon as
the results are available, it is possible to mark some of them as examples. The
collection of results is then reordered according to the similarity with the examples.
By rearranging the results this way, we can avoid the need for manual inspection of each result. Moreover the
similarity-based arranging of search results has an extra advantage: the user does not need
to have an appropriate example.

\section{Preliminaries}

\subsection{Fuzzy Sets}

A fuzzy set $A$ in a universe $X$ is a $X-[0,1]$ mapping that associates with each element
$x$ from the universe $X$ a degree of membership $A(x)$ \cite{kerre:vaagmodellen}. 
We use the notation ${\mathcal{F}}(X)$ for the class of fuzzy sets in $X$. 

For two fuzzy 
sets $A$ and $B$ in $X$, the classical set theoretic operations $co$ (complement), $\cap$ 
(intersection) and $\cup$ (union) can be generalized as follows 
\cite{kerre:vaagmodellen}: 
$(co\,A)(x)=1-A(x)$, $(A \cap B)(x)=\min\{A(x),B(x)\}$ and 
$(A \cup B)(x)=\max\{A(x),B(x)\}$ for each $x \in X$. The support of a fuzzy set
$A$ in $X$ is given by $supp\,A = \{x\in X \mid A(x) > 0\}$. For a fuzzy set $A$ in $X$ with
finite support, the sigma count, defined as $|A|=\sum_{x \in X} A(x)$, is a generalisation 
of the crisp concept cardinality.

Fuzzy sets can be further generalized to $L$-fuzzy sets. Such a $L$-fuzzy set in a universe
$X$ is a $X-L$ mapping, with $(L,\le)$ a complete lattice for some partial order relation 
$\le$ \cite{kerre:vaagmodellen}.  

\subsection{Fuzzy Similarity Measures}

Fuzzy similarity measures are
fuzzy sets in ${\mathcal{F}}(X) \times {\mathcal{F}}(X)$ that express the similarity between
each pair of fuzzy sets in $X$, i.e. the degree of membership $M(A,B)$ of 
$(A,B) \in ({\mathcal{F}}(X))^{2}$ denotes the similarity between $A$ and $B$. If $A$ and $B$
are completely similar then $M(A,B)=1$, otherwise $M(A,B)<1$. The fuzzy similarity measures
that we considered are a selection from the measures discussed in 
\cite{vanderweken:similariteitsmaten}.

\section{Similarity Measures for Images}

\subsection{Construction and Properties}

If we are able to identify objects with fuzzy sets, then we can use fuzzy similarity
measures to compare these objects. We have used this idea to construct several similarity 
measures for images. All these measures are reflexive and symmetric, i.e. each measure $M$ 
satisfies $M(A,A)=1$ and $M(A,B)=M(B,A)$ for every pair of images $(A,B)$.

\subsection{Restricting the Execution Time}

When identifying images with fuzzy sets, it can happen that the supports of the
sets contain far less elements than the universe. In such cases we can restrict the
execution time dra\-ma\-ti\-cal\-ly by rewriting the fuzzy similarity measures. Consider for instance
$M_5$ \cite{vanderweken:similariteitsmaten}: 
\begin{displaymath}
M_{5}(A,B) = \frac{\min\{|A|,|B|\}}{\max\{|A|,|B|\}} = \frac{\min\{||A||,||B||\}}{\max\{||A||,||B||\}}
\end{displaymath}
where $||C||=\sum_{x\in X'}C(x)$ with $C \in {\mathcal{F}}(X)$ and $X'=supp\,A \cup supp\,B$.
The old form requires $2 \cdot (|X|-1)$ additions to calculate $|A|$ and $|B|$, 
followed by one comparison to determine $\min\{|A|,|B|\}$ and $\max\{|A|,|B|\}$ and finally 
one division. For the rewritten form we need $ 2 \cdot (|X'|-1)$ additions, one comparison
and one division. Hence the calculation of the new form will indeed be faster, since 
$|X'| \ll |X|$ implies that $2 \cdot (|X'|-1) + 2 \ll 2 \cdot (|X|-1) + 2$. 
Furthermore the execution times are the same if $X'=X$, which means that the new form is
always at least as fast as the old one.

\subsection{Pixel-based Similarity Measures}

A digital greyscale image can be identified with a fuzzy set in a straightforward manner:
\begin{displaymath}
\begin{array}{rcl}
A(p) = 1 & \iff & p \text{ is a white pixel} \\
A(p) = 0 & \iff & p \text{ is a black pixel} \\
A(p) \in\ ]0,1[ & \iff & p \textrm{ is a greyscale pixel}
\end{array}
\end{displaymath}
for all $p \in P$, with $P$ the universe of pixels ($P \subset {\mathbb{N}}^{2}$). Thereby 
applies: the greater the greyscale value, the lighter the pixel. We call such measures 
pixel-based, because the universe consists of pixels.

On the internet, however, most images are colour images. We consider three approaches to 
construct similarity measures for colour images: (1) convert the images to greyscale 
images and use the aforementioned representation, (2) interpret the images as three 
greyscale images --- one for each colour component --- and use an aggregation
operator \cite{detyniecki:numerical_aggregation_operators} to combine the three resulting
similarities, and (3) identify the 
images with $[0,1]^3$-fuzzy sets  and generalize the fuzzy measures in order to make them 
applicable to these $[0,1]^3$-fuzzy sets
\cite{vanderweken:construction_of_quality_measures}. 

Another problem that needs to be solved, is that the measures need to be able to cope with
images of different resolutions. Measures based on the preceding representations lack that 
feature because the universe of the resulting fuzzy set depends on the resolution of the
image, and only fuzzy sets in the same universe can be compared using a fuzzy similarity 
measure. We can get rid of this shortcoming by constructing similarity measures that
use intermediate images of the same resolution, e.g.~rescaled versions of the 
original images. An alternative solution consists of partitioning the images in parts
of equal resolution, calculating the similarity between certain parts and aggregating
the resulting similarities to one value \cite{detyniecki:numerical_aggregation_operators}.

\subsection{Colour-based Similarity Measures}

Colour-based similarity measures for images use fuzzy sets in a universe that contains colours 
instead of pixels. They can be constructed by considering image histograms. We define
the histogram $h_A$ for an image $A$ as 
\begin{displaymath}
h_A(c) = \sum_{p \in P} \delta (bin(c) - bin(A(p))) 
\end{displaymath} 
for or all $c \in C' \subset C$, with $C$ the universe of colours, $\delta$ the Dirac function ($\delta (0)=1$ and $\delta (x)=0$ for all 
$x \in {\mathbb{Z}} \setminus \{0\}$) and $bin$ a mapping from $C$ to $\{1,2,\ldots,|C'|\}$.
The histogram $h_A$ for an image $A$ can be converted to a fuzzy set by normalizing it 
\cite{vanderweken:similariteitsmaten,vanderweken:construction_of_quality_measures,vertan:fuzzy_histograms}:
\begin{displaymath}
H_A(c) = \frac{\displaystyle h_A(c)}{\displaystyle \max_{c \in C'}(h_A(c))}
\end{displaymath}
for all $c \in C'$. We call the fuzzy sets obtained in this way 
pseudo-fuzzy histograms.

In addition to the pseudo-fuzzy histograms, which are fuzzy interpretations of the
classical histogram, we also consider a fuzzy histogram. We define the fuzzy histogram
$\widetilde{H}_A$ for an image $A$ as follows \cite{vertan:fuzzy_histograms}:
%\begin{displaymath} 
$\widetilde{H}_A = \bigcup_{c \in C_A} \widetilde{C}_c$
%\end{displaymath}
where $C_A$ is the set consisting of the colours appearing in $A$ and 
\begin{displaymath}
\widetilde{C}_c(c') =
\begin{cases}
1 & \text{if } d'_{c,c'} = \frac{d \left( lab(c), lab(c') \right)}{2.3} \leq 1 \\ 
\exp \frac{- \left( d'_{c,c'} - 1 \right)^2}{2 \cdot \lambda^2} & \text{else}
\end{cases}
\end{displaymath} 
for all $c' \in C$, with $d$ the Euclidian distance, $lab$ the function that maps each 
colour to its L*a*b*-coordinates and $\lambda$ an adjustable parameter.

Various extensions of the classical histogram have been proposed in the literature 
(e.g. smoothed histograms \cite{vertan:fuzzy_histograms}).
Many of these extensions are special cases of the extended histogram, which we define
as follows:
\begin{displaymath}
\mathring{h}_A(c) = \sum_{c' \in C'} \sum_{p \in P} \delta (bin(c')-bin(A(p))) \cdot w_{A,c}(p)
\end{displaymath}
for each $c$ in $C'$. If 
$w_{A,c}(p)=\delta(bin(c)-bin(A(p)))$, $\forall p \in P$, then $\mathring{h}_A=h_A$.
Hence the classical histogram is also a special case of the extended 
histogram.

\subsection{Experimental Observations}

We have constructed and implemented several pixel-based and colour-based similarity measures 
for images. In order to test and compare all these measures, we used each measure to 
arrange a subcollection of the Columbia Object Image Library (COIL) \cite{coil-100} 
according to the similarity with an example. Then we
evaluated the resulting orderings using the Normalized Average Rank (NAR) \cite{muller:perf_eval}.
For every measure we calculated multiple NARs, each corresponding to another example.
We call the arithmetic mean of the NARs for a certain measure the Global Normalized 
Average Rank (GNAR) for that measure. The lower the GNAR, the better the performance
of the corresponding measure.

In our experiments we observed very good results for the similarity measure that employs
$M_{I_3}$ \cite{vanderweken:similariteitsmaten}
for comparing pseudo-fuzzy histograms in the Irb colour space ($I = (R+G+B)/3$, 
$r=R/(3\cdot I)$ and $b=B/(3\cdot I)$) using
\begin{displaymath}
bin(c_1,c_2,c_3)=\left[ \sum_{i=1}^3 \left( \prod_{j=1}^{i-1} N_j \right) \mathit{fl}(c_i, N_i) \right] 
+ 1
\end{displaymath}
for all $(c_1,c_2,c_3) \in C$, with
$N_1=4$, $N_2=N_3=8$ and
\begin{displaymath}
\mathit{fl}(x,n) = \begin{cases}
\lfloor x \cdot n \rfloor & \text{if } x < 1 \\ 
n - 1 & \text{if } x = 1
\end{cases}
\end{displaymath}
for each $x \in [0,1]$ and $n \in \mathbb{N}$.

\section{Conclusions}
Fuzzy set theory proves to be very convenient for the mo\-del\-ling of similarity measures
for images. Such measures can be used as a basis for an extension of traditional TBIR
that incorporates content-based aspects without sacrificing the scalability of the system.
Our prototype that implements this extension is available at 
\url{http://imilarity.berlios.de}. 

\nocite{*}
\bibliographystyle{phdsymp}
\bibliography{artikel} 


\end{document}

