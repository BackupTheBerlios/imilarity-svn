\chapter{Inleiding}

Op dit eigenste moment zijn er waarschijnlijk enkele honderden \emph{spiders} actief op internet.
Deze computerprogramma's, die soms ook \emph{robots} of \emph{wanderers} worden genoemd, reizen
het internet rond om bepaalde documenten -- in het bijzonder afbeeldingen -- te localiseren. Deze 
documenten worden ge\"indexeerd in een databank, die dan doorzocht kan worden door een zoekmachine. 

Zoekmachines zoals \emph{Google}\footnote{\url{http://www.google.com}} en 
\emph{Yahoo}\footnote{\url{http://www.yahoo.com}} hebben op die manier reeds databanken
opgebouwd die meer dan een miljard afbeeldingen bevatten. Het wordt bijgevolg steeds belangrijker
om manieren te zoeken om deze gigantische collecties van afbeeldingen op een effeci\"ente wijze
te doorzoeken.


\section{Text-based image retrieval}

Nagenoeg alle bestaande zoekmachines bieden \emph{text-based image retrieval} (TBIR) aan. Hierbij
wordt elke afbeelding voorzien van tekstuele annotaties, zoals bijvoorbeeld de 
bestandsnaam of woorden uit de webpagina waarin de afbeelding gebruikt wordt. Deze annotaties
kunnen dan gebruikt worden voor het indexeren van de afbeeldingen in de databank.


\section{Content-based image retrieval}

Omdat de tekst-gebaseerde aanpak in de praktijk dikwijls tekort schiet, is men op zoek gegaan 
naar manier om het zoeken te baseren op de visuele inhoud van de afbeeldingen. Bij 
\emph{content-based image retrieval} (CBIR) maakt men gebruik van een proces dat 
\emph{(visual) feature extraction} genoemd wordt. Dit proces zet een afbeelding om in een 
\emph{feature vector}. Met behulp van multidimensionale indexering kan men deze kenmerkenvector
dan gebruiken als alternatief voor de tekstuele annotaties bij TBIR.


\section{Similariteitsgebaseerd rangschikken}

Door hun grote complexiteit is het vrijwel onmogelijk om \emph{content-based image retrial systems} 
(CBIRSs) te gebruiken voor het doorzoeken van zeer omvangrijke databanken. 