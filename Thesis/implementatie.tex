\chapter{Implementatie}

De belangrijkste bijdrage van dit eindwerk is de bouw van een prototype
dat het similariteitsgebaseerd rangschikken van zoekresultaten implementeert.
Dat prototype wordt in dit hoofstuk in grote lijnen besproken. Voor de details
verwijzen we naar de broncode op de bijgevoegde CD-ROM.

\section{Imilarity}

Aan de basis van het prototype ligt \defin{Imilarity}, een collectie van Java klassen 
die kunnen gebruikt worden om beelden op te halen en vervolgens te vergelijken. Een 
dergelijke collectie van nauw samenwerkende klassen wordt een \defin{framework} genoemd. 

Het ophalen van beelden gebeurt aan de hand van een klasse die de interface 
{\klassefont Providor} implementeert. Codefragment~\ref{code:providor} toont de definitie
van die interface. Het is duidelijk dat de beelden in groepjes van 
{\klassefont getPageSize()} doorgegeven worden, in plaats van allemaal in \'e\'en keer. 
Dat heeft als voordeel dat de vooruitgang van het ophalen van de beelden kan 
opgevolgd worden. E\'en groepje beelden noemen we een \defin{pagina}.

\begin{code}[bp]
\vspace{5pt}
\begin{lgrind}
\input{code/Providor.java.tex}
\end{lgrind}
\vspace{5pt}
\caption{\label{code:providor}Definitie van de {\klassefont Providor} interface.}
\end{code}

Zodra de beelden opgehaald zijn, kunnen ze met elkaar vergeleken worden. Daarvoor
gebruiken we de interface {\klassefont ImageMeasure} uit 
codefragment~\ref{code:imagemeasure}. Een klasse die {\klassefont ImageMeasure} 
implementeert, komt overeen met een similariteitsmaat voor beelden. Voor de 
vaagsimilariteitsmaten gebruiken we de interface {\klassefont FuzzyMeasure}.
Codefragment~\ref{code:fuzzymeasure} bevat de definitie van die interface.
Beide interfaces zijn uitbreidingen van de interface {\klassefont Measure}
uit codefragment~\ref{code:measure}. 
Een object van het type {\klassefont FuzzyImageMeasure} vergelijkt beelden
met behulp van een {\klassefont FuzzyMeasure}-object. Alle overige
{\klassefont ImageMeasure}-objecten maken gebruiken van een ander 
{\klassefont ImageMeasure}-object. Zo zal een {\klassefont ComponentsImageMeasure}
bijvoorbeeld een andere {\klassefont ImageMeasure} gebruiken om de grijswaardebeelden
van elk van de kleurcomponenten met elkaar te vergelijken. 

\begin{code}[!b]
\vspace{5pt}
\begin{lgrind}
\input{code/ImageMeasure.java.tex}
\end{lgrind}
\vspace{5pt}
\caption{\label{code:imagemeasure}Definitie van de interface {\klassefont ImageMeasure}.}
\end{code}
%
\begin{code}[!b]
\vspace{5pt}
\begin{lgrind}
\input{code/FuzzyMeasure.java.tex}
\end{lgrind}
\vspace{5pt}
\caption{\label{code:fuzzymeasure}Definitie van de interface {\klassefont FuzzyMeasure}.}
\end{code}
%
\begin{code}[!b]
\vspace{5pt}
\begin{lgrind}
\input{code/Measure.java.tex}
\end{lgrind}
\vspace{5pt}
\caption{\label{code:measure}Definitie van de interface {\klassefont Measure}.}
\end{code}

Tot nu toe hebben we, bij het bepalen van de rangschikkingen, telkens \'e\'en 
voorbeeld gebruikt. Het is echter zeker niet ondenkbaar dat een gebruiker
van het prototype meerdere voorbeelden wenst op te geven. Om dat toe te laten,
maken we gebruik van aggregatieoperatoren. We berekenen eerst de similariteit met
elk van de voorbeelden, om daarna dan die similariteiten te aggregeren tot \'e\'en
getal met behulp van een aggregatieoperator. De huidige implementatie 
ondersteunt het gemiddelde, het maximum, het minimum, het product en de 
algebra\"ische som. Het zou echter interessant zijn om na te gaan of er betere
resultaten kunnen bekomen worden door meer geavanceerde operatoren te gebruiken.

De klasse {\klassefont Imilarity} abstraheert het ophalen en rangschikken van
beelden. Codefragement~\ref{code:simple_example} bevat een zeer eenvoudig 
\begin{code}[bp]
\vspace{5pt}
\begin{lgrind}
\input{code/SimpleExample.java.tex}
\end{lgrind}
\vspace{5pt}
\caption{\label{code:simple_example}Simpel voorbeeldprogramma dat het gebruik van
de klasse {\klassefont Imilarity} illustreert.}
\end{code}
voorbeeldprogramma dat het gebruik van die klasse illustreert. Het commando
\texttt{java SimpleExample zadeh} genereert de volgende uitvoer:
\begin{verbatim}
Loading images ####################
Reordering images ####################
1: 'Hemati Zadeh'
2: 'zadeh'
\end{verbatim} 
Dit zijn de verschillende stappen die daarbij plaatsvinden:
\begin{enumerate}
  \item Het {\klassefont Imilarity}-object wordt aangemaakt.
  \item De similariteitsmaat wordt ingesteld via de methode {\klassefont setMeasure}.
  \item Met behulp van de methode {\klassefont setAggregator} wordt het maximum ingesteld als
		aggregatieoperator. 
  \item Als {\klassefont Providor} wordt er een {\klassefont YahooProvidor} ingesteld.
  	    Het ligt voor de hand dat die providor gebruik maakt van
  	    \emph{Yahoo Image Search}. Bij de constructie van een dergelijke providor
  	    moet er een zoekterm worden opgegeven. In dit geval wordt daarvoor het eerste 
  	    commandolijn-argument gebruikt. Vervolgens worden de pagina's in 
  	    het geheugen geladen door {\klassefont loadPage} op te roepen voor elke pagina.
  \item De eerste twee zoekresultaten worden opgegeven als voorbeeld en de pagina's
        worden geordend volgens similariteit met die voorbeelden. Daarna wordt
        {\klassefont mergeReorderedPages} uitgevoerd. Zoals de naam al doet vermoeden,
        zorgt die methode ervoor dat de geordende pagina's samengevoegd 
        worden tot een globale rangschikking.
  \item Tenslotte worden de namen van de eerste twee beelden in de rangschikking 
  	    uitgeschreven.
\end{enumerate}

\section{Imperforate}

In de voorgaande hoofdstukken hebben we de geconstrueerde similariteitsmaten 
met elkaar vergeleken door de GGGR-waarden te berekenen voor een bepaalde testcollectie.
\defin{Imperforate} is een programma dat toelaat om dergelijke experimenten 
uit te voeren. Het kan gestart worden door dubbel te klikken op het 
bestand ``imperforate.jar'' op de bijgevoegde CD-ROM. Eventueel kan het ook
manueel opgestart worden via het commando \texttt{java -jar imperforate.jar}. 
In beide gevallen wordt Imperforate uitgevoerd met de standaard maximale grootte
van het grabbelgeheugen. Voor sommige testcollecties volstaat dat echter niet.
Het kan bijgevolg nuttig zijn om het maximale geheugenverbruik te 
vergroten. Als we bijvoorbeeld het commando 
\texttt{java -Xmx256M -jar imperforate.jar} gebruiken,
dan wordt de maximale grootte van het grabbelgeheugen verhoogd tot 256 MB. 

Figuur~\ref{fig:imperforate_1} toont het initi\"ele venster van Imperforate. Vooraleer er iets
kan berekend worden, moet er eerst een beeldencollectie ingeladen worden.
Dat kan door op de knop ``Add ...'' te klikken, waardoor het venster uit 
figuur~\ref{fig:imperforate_2} verschijnt. Via dat venster kan er een 
``*.class''-bestand opgegeven worden. Dat bestand moet een gecompileerde 
Java-klasse zijn, die de interface {\klassefont ImageCollection} uit 
codefragment~\ref{code:imagecollection} implementeert. De map ``image\_collections''
op de CD-ROM bevat drie dergelijke bestanden. E\'en voor de COIL subcollectie,
\'e\'en voor een selectie van beelden uit de UCID collectie \cite{ucid} en
een derde voor een deel van de collectie van Kato 
\cite{kato:cbir_using_stochastic_paintbrush}.

\begin{code}[!bp]
\vspace{5pt}
\begin{lgrind}
\input{code/ImageCollection.java.tex}
\end{lgrind}
\vspace{5pt}
\caption{\label{code:imagecollection}Definitie van de interface 
{\klassefont ImageCollection}.}
\end{code}

\begin{figure}[bp]
\vspace{6pt}
\centering
\begin{tabular}{@{}c@{}c@{}}
\multicolumn{2}{c}{\subfigure[] {
\begin{minipage}{\textwidth}
\centering
\includegraphics[height=8cm]{images/imperforate_1.eps}
\vspace{6pt}
\end{minipage}
\label{fig:imperforate_1}
}}\\
\multicolumn{2}{c}{\subfigure[] {
\begin{minipage}{\textwidth}
\centering
\includegraphics[height=8cm]{images/imperforate_3.eps}
\vspace{6pt}
\end{minipage}
\label{fig:imperforate_3}
}}\\
\subfigure[] {
\begin{minipage}{0.48\textwidth}
\centering
\includegraphics[height=5cm]{images/imperforate_2.eps}
\vspace{6pt}
\end{minipage}
\label{fig:imperforate_2}
}
&
\subfigure[] {
\begin{minipage}{0.48\textwidth}
\centering
\includegraphics[height=5cm]{images/imperforate_4.eps}
\vspace{6pt}
\end{minipage}
\label{fig:imperforate_4}
}
\end{tabular}\vspace{3pt}
\caption{\label{fig:imperforate}Enkele screenshots van Imperforate.}
\end{figure}

Zodra er een beeldencollectie is ingeladen, kunnen de GGGRs berekend worden.
Daartoe moet er op ``Calculate'' geklikt worden nadat de gewenste 
similariteitsmaten geselecteerd werden. De waarden in de kolommen worden dan 
geleidelijk aan ingevuld. Figuur~\ref{fig:imperforate_3} illustreert dat de 
inhoud van een kolom kan gesorteerd worden door op de titel van die kolom te klikken.
Wanneer de gebruiker op ``Show results'' klikt, dan worden de resultaten van de
geselecteerde maten getoond in een venster zoals dat uit 
figuur~\ref{fig:imperforate_4}. De overblijvende knop ``Print gnuplot data'' 
zorgt ervoor dat
de gegevens van de geselecteerde maten uitgeschreven worden in een formaat dat
rechtstreeks kan gebruikt worden als invoer voor het programma \emph{gnuplot}. 
Op die manier kunnen de gegenereerde data gemakkelijk omgezet worden naar een grafiek.


\section{Giggle}

Het eigenlijke prototype noemen we \defin{Giggle}. Uiteraard is Giggle, net zoals
Imperforate, gebaseerd op het Imilarity framework. Het opstarten gebeurt
op dezelfde manier als bij Imperforate, maar dan met het bestand ``giggle.jar''
in plaats van ``imperforate.jar''.

Figuur~\ref{fig:giggle_1} toont
hoe het hoofdvenster van Giggle er uitziet wanneer er zoekresultaten worden
ingeladen. In dit geval werd de zoekterm ``zadeh'' gebruikt. 
Door op \'e\'en van
de zoekresultaten te klikken en vervolgens ``Add to examples'' te kiezen,
kunnen er voorbeelden worden toegevoegd. Daarna kan de gebruiker dan, met behulp 
van de knop ``Reorder results'', de opdracht geven om de zoekresultaten te herordenen 
volgens similariteit met de voorbeelden. In figuur~\ref{fig:giggle_2} wordt het
resultaat van een dergelijke herordening getoond. Zoals weergegeven in
figuur~\ref{fig:giggle_3}, kunnen de gevonden beelden op ware grootte bekeken worden via de link 
``Show full size image''. De gebruiker kan een beeld ook opslaan op zijn of haar 
lokale schijf door te klikken op de grote weergave.

De knoppen ``Use first as examples'' en ``Clear examples'' spreken voor zich. Via
de knop ``Settings ...'' kunnen de vensters uit figuur~\ref{fig:giggle_4}
geopend worden. Die vensters laten toe om een andere similariteitsmaat en/of
aggregatieoperator in te stellen. Standaard gebruikt Giggle de similariteitsmaat 
die pseudo-vage Irb-histogrammen vergelijkt met behulp van $M_{I_3}$. 
Die maat gaf immers een zeer lage GGGR-waarde bij onze experimenten. Het maximum is
de standaard aggregatieoperator.

\begin{figure}[bp]
\vspace{6pt}
\centering
\begin{tabular}{@{}c@{}c@{}}
\multicolumn{2}{c}{\subfigure[] {
\begin{minipage}{\textwidth}
\centering
\includegraphics[height=8cm]{images/giggle_1.eps}
\vspace{6pt}
\end{minipage}
\label{fig:giggle_1}
}}\\
\multicolumn{2}{c}{\subfigure[] {
\begin{minipage}{\textwidth}
\centering
\includegraphics[height=8cm]{images/giggle_2.eps}
\vspace{6pt}
\end{minipage}
\label{fig:giggle_2}
}}\\
\subfigure[] {
\begin{minipage}{0.48\textwidth}
\centering
\includegraphics[height=5cm]{images/giggle_3.eps}
\vspace{6pt}
\end{minipage}
\label{fig:giggle_3}
}
&
\subfigure[] {
\begin{minipage}{0.48\textwidth}
\centering
\includegraphics[height=5cm]{images/giggle_4.eps}
\vspace{6pt}
\end{minipage}
\label{fig:giggle_4}
}
\end{tabular}\vspace{3pt}
\caption{\label{fig:giggle}Enkele screenshots van Giggle.}
\end{figure}


\section{Enkele voorbeelden}
\label{sectie:tests_banaan}

Tot slot van dit hoofdstuk bekijken we enkele rangschikkingen die door Giggle
gegenereerd werden voor de zoekresultaten van de zoekterm ``banaan''. We
gebruiken daarbij de beelden uit figuur~\ref{fig:tests_banaan_voorbeelden} als
voorbeelden. Het aggregeren van de similariteiten voor elk van die voorbeelden,
doen we telkens met behulp van het maximum. We gaan dus met andere woorden op
zoek naar beelden die similair zijn met minstens \'e\'en van de voorbeelden. Als
we het minimum -- of een andere t-norm -- zouden gebruiken, dan zouden we eisen
dat de beelden similair zijn met \emph{alle} voorbeelden. Bij de componentsgewijze
similariteitsmaten gebruiken we het gemiddelde om de similariteiten van de 
verschillende componenten te aggregeren.

Figuur~\ref{fig:tests_banaan_1} toont de eerste tien beelden van drie rangschikkingen
die werden bekomen aan de hand van pixelgebaseerde similariteitsmaten. Voor de
rangschikking in figuur~\ref{fig:tests_banaan_2} en figuur~\ref{fig:tests_banaan_3}
werden kleurgebaseerde maten gebruikt. Het valt op dat het glad SCT-histogram in dit
geval betere resultaten geeft dan het Irb-histogram. Verder zien we ook dat
het glad spatiaal SCT-histogram hier minder goede resultaten geeft, terwijl het
glad SCT-randhistogram wel vrij behoorlijk presteert.

\begin{figure}[bp]
\vspace{6pt}
\centering
\begin{tabular}{@{}ccc@{}}
\includegraphics[scale=0.6]{images/banaan_irb_1_1.eps} &
\includegraphics[scale=0.6]{images/banaan_irb_1_2.eps} &
\includegraphics[scale=0.6]{images/banaan_irb_1_3.eps}
\end{tabular}
\vspace{5pt}
\caption{\label{fig:tests_banaan_voorbeelden}De voorbeelden die we gebruiken bij het
rangschikken van de zoekresultaten voor ``banaan''.}
\end{figure}

\begin{figure}[bp]
\subfigure[]{
\begin{minipage}{\textwidth}
\vspace{6pt}
\centering
\begin{tabular}{@{}ccccc@{}}
\includegraphics[scale=0.6]{images/banaan_scaling_1_1.eps} &
\includegraphics[scale=0.6]{images/banaan_scaling_1_2.eps} &
\includegraphics[scale=0.6]{images/banaan_scaling_1_3.eps} &
\includegraphics[scale=0.6]{images/banaan_scaling_1_4.eps} &
\includegraphics[scale=0.6]{images/banaan_scaling_1_5.eps} \\
\includegraphics[scale=0.6]{images/banaan_scaling_2_1.eps} & 
\includegraphics[scale=0.6]{images/banaan_scaling_2_2.eps} &
\includegraphics[scale=0.45]{images/banaan_scaling_2_3.eps} &
\includegraphics[scale=0.6]{images/banaan_scaling_2_4.eps} & 
\includegraphics[scale=0.6]{images/banaan_scaling_2_5.eps}
\end{tabular}
\vspace{5pt}
\end{minipage}
}\\
\subfigure[]{
\begin{minipage}{\textwidth}
\vspace{6pt}
\centering
\begin{tabular}{@{}ccccc@{}}
\includegraphics[scale=0.6]{images/banaan_moments_1_1.eps} &
\includegraphics[scale=0.6]{images/banaan_moments_1_2.eps} &
\includegraphics[scale=0.6]{images/banaan_moments_1_3.eps} &
\includegraphics[scale=0.6]{images/banaan_moments_1_4.eps} &
\includegraphics[scale=0.6]{images/banaan_moments_1_5.eps} \\
\includegraphics[scale=0.6]{images/banaan_moments_2_1.eps} & 
\includegraphics[scale=0.6]{images/banaan_moments_2_2.eps} &
\includegraphics[scale=0.6]{images/banaan_moments_2_3.eps} &
\includegraphics[scale=0.6]{images/banaan_moments_2_4.eps} & 
\includegraphics[scale=0.45]{images/banaan_moments_2_5.eps}
\end{tabular}
\vspace{5pt}
\end{minipage}
}\\
\subfigure[]{
\begin{minipage}{\textwidth}
\vspace{6pt}
\centering
\begin{tabular}{@{}ccccc@{}}
\includegraphics[scale=0.6]{images/banaan_parts_1_1.eps} &
\includegraphics[scale=0.6]{images/banaan_parts_1_2.eps} &
\includegraphics[scale=0.6]{images/banaan_parts_1_3.eps} &
\includegraphics[scale=0.6]{images/banaan_parts_1_4.eps} &
\includegraphics[scale=0.6]{images/banaan_parts_1_5.eps} \\
\includegraphics[scale=0.6]{images/banaan_parts_2_1.eps} & 
\includegraphics[scale=0.6]{images/banaan_parts_2_2.eps} &
\includegraphics[scale=0.6]{images/banaan_parts_2_3.eps} &
\includegraphics[scale=0.6]{images/banaan_parts_2_4.eps} & 
\includegraphics[scale=0.6]{images/banaan_parts_2_5.eps}
\end{tabular}
\vspace{5pt}
\end{minipage}
}
\caption{\label{fig:tests_banaan_1}De eerst tien beelden van de gerangschikte zoekresultaten voor ``banaan'' 
op basis van componentsgewijze toepassing van (a) $M_7$, (b) $M_9$ en (c) $M_{I_{3c}}$, in combinatie met (a) herschaling, (b) kleurmomenten en (c) beeldonderdelen.}
\end{figure}

\begin{figure}[bp]
\subfigure[]{
\begin{minipage}{\textwidth}
\vspace{6pt}
\centering
\begin{tabular}{@{}ccccc@{}}
\includegraphics[scale=0.6]{images/banaan_irb_1_1.eps} &
\includegraphics[scale=0.6]{images/banaan_irb_1_2.eps} &
\includegraphics[scale=0.6]{images/banaan_irb_1_3.eps} &
\includegraphics[scale=0.6]{images/banaan_irb_1_4.eps} &
\includegraphics[scale=0.6]{images/banaan_irb_1_5.eps} \\
\includegraphics[scale=0.6]{images/banaan_irb_2_1.eps} & 
\includegraphics[scale=0.6]{images/banaan_irb_2_2.eps} &
\includegraphics[scale=0.6]{images/banaan_irb_2_3.eps} &
\includegraphics[scale=0.6]{images/banaan_irb_2_4.eps} & 
\includegraphics[scale=0.6]{images/banaan_irb_2_5.eps}
\end{tabular}
\vspace{5pt}
\end{minipage}
}\\
\subfigure[]{
\begin{minipage}{\textwidth}
\vspace{6pt}
\centering
\begin{tabular}{@{}ccccc@{}}
\includegraphics[scale=0.6]{images/banaan_sct_1_1.eps} &
\includegraphics[scale=0.6]{images/banaan_sct_1_2.eps} &
\includegraphics[scale=0.6]{images/banaan_sct_1_3.eps} &
\includegraphics[scale=0.6]{images/banaan_sct_1_4.eps} &
\includegraphics[scale=0.6]{images/banaan_sct_1_5.eps} \\
\includegraphics[scale=0.6]{images/banaan_sct_2_1.eps} & 
\includegraphics[scale=0.6]{images/banaan_sct_2_2.eps} &
\includegraphics[scale=0.6]{images/banaan_sct_2_3.eps} &
\includegraphics[scale=0.6]{images/banaan_sct_2_4.eps} & 
\includegraphics[scale=0.6]{images/banaan_sct_2_5.eps}
\end{tabular}
\vspace{5pt}
\end{minipage}
}\\
\subfigure[]{
\begin{minipage}{\textwidth}
\vspace{6pt}
\centering
\begin{tabular}{@{}ccccc@{}}
\includegraphics[scale=0.6]{images/banaan_smoothed_sct_1_1.eps} &
\includegraphics[scale=0.6]{images/banaan_smoothed_sct_1_2.eps} &
\includegraphics[scale=0.6]{images/banaan_smoothed_sct_1_3.eps} &
\includegraphics[scale=0.6]{images/banaan_smoothed_sct_1_4.eps} &
\includegraphics[scale=0.6]{images/banaan_smoothed_sct_1_5.eps} \\
\includegraphics[scale=0.6]{images/banaan_smoothed_sct_2_1.eps} & 
\includegraphics[scale=0.6]{images/banaan_smoothed_sct_2_2.eps} &
\includegraphics[scale=0.6]{images/banaan_smoothed_sct_2_3.eps} &
\includegraphics[scale=0.6]{images/banaan_smoothed_sct_2_4.eps} & 
\includegraphics[scale=0.6]{images/banaan_smoothed_sct_2_5.eps}
\end{tabular}
\vspace{5pt}
\end{minipage}
}
\caption{\label{fig:tests_banaan_2}De eerst tien beelden van de gerangschikte zoekresultaten voor ``banaan'' 
op basis van $M_{I_3}$ in combinatie met (a) het Irb-histogram, (b) het SCT-histogram en (c) het glad SCT-histogram.}
\end{figure}

\begin{figure}[bp]
\subfigure[]{
\begin{minipage}{\textwidth}
\vspace{6pt}
\centering
\begin{tabular}{@{}ccccc@{}}
\includegraphics[scale=0.6]{images/banaan_smoothed_sct_1_1.eps} &
\includegraphics[scale=0.6]{images/banaan_smoothed_sct_1_2.eps} &
\includegraphics[scale=0.6]{images/banaan_smoothed_sct_1_3.eps} &
\includegraphics[scale=0.6]{images/banaan_smoothed_sct_1_4.eps} &
\includegraphics[scale=0.6]{images/banaan_smoothed_sct_1_5.eps} \\
\includegraphics[scale=0.6]{images/banaan_smoothed_sct_2_1.eps} & 
\includegraphics[scale=0.6]{images/banaan_smoothed_sct_2_2.eps} &
\includegraphics[scale=0.6]{images/banaan_smoothed_sct_2_3.eps} &
\includegraphics[scale=0.6]{images/banaan_smoothed_sct_2_4.eps} & 
\includegraphics[scale=0.6]{images/banaan_smoothed_sct_2_5.eps}
\end{tabular}
\vspace{5pt}
\end{minipage}
}\\
\subfigure[]{
\begin{minipage}{\textwidth}
\vspace{6pt}
\centering
\begin{tabular}{@{}ccccc@{}}
\includegraphics[scale=0.6]{images/banaan_smoothed_sct_spatial_1_1.eps} &
\includegraphics[scale=0.6]{images/banaan_smoothed_sct_spatial_1_2.eps} &
\includegraphics[scale=0.6]{images/banaan_smoothed_sct_spatial_1_3.eps} &
\includegraphics[scale=0.6]{images/banaan_smoothed_sct_spatial_1_4.eps} &
\includegraphics[scale=0.6]{images/banaan_smoothed_sct_spatial_1_5.eps} \\
\includegraphics[scale=0.6]{images/banaan_smoothed_sct_spatial_2_1.eps} & 
\includegraphics[scale=0.6]{images/banaan_smoothed_sct_spatial_2_2.eps} &
\includegraphics[scale=0.6]{images/banaan_smoothed_sct_spatial_2_3.eps} &
\includegraphics[scale=0.6]{images/banaan_smoothed_sct_spatial_2_4.eps} & 
\includegraphics[scale=0.6]{images/banaan_smoothed_sct_spatial_2_5.eps}
\end{tabular}
\vspace{5pt}
\end{minipage}
}\\
\subfigure[]{
\begin{minipage}{\textwidth}
\vspace{6pt}
\centering
\begin{tabular}{@{}ccccc@{}}
\includegraphics[scale=0.6]{images/banaan_smoothed_sct_edgy_1_1.eps} &
\includegraphics[scale=0.6]{images/banaan_smoothed_sct_edgy_1_2.eps} &
\includegraphics[scale=0.6]{images/banaan_smoothed_sct_edgy_1_3.eps} &
\includegraphics[scale=0.6]{images/banaan_smoothed_sct_edgy_1_4.eps} &
\includegraphics[scale=0.6]{images/banaan_smoothed_sct_edgy_1_5.eps} \\
\includegraphics[scale=0.6]{images/banaan_smoothed_sct_edgy_2_1.eps} & 
\includegraphics[scale=0.6]{images/banaan_smoothed_sct_edgy_2_2.eps} &
\includegraphics[scale=0.6]{images/banaan_smoothed_sct_edgy_2_3.eps} &
\includegraphics[scale=0.6]{images/banaan_smoothed_sct_edgy_2_4.eps} & 
\includegraphics[scale=0.6]{images/banaan_smoothed_sct_edgy_2_5.eps}
\end{tabular}
\vspace{5pt}
\end{minipage}
}
\caption{\label{fig:tests_banaan_3}De eerst tien beelden van de gerangschikte zoekresultaten voor ``banaan'' 
op basis van $M_{I_3}$ in combinatie met (a) het glad SCT-histogram, (b) het glad spatiaal SCT-histogram en 
(c) het glad SCT-randhistogram.}
\end{figure}