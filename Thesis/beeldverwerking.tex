\chapter{Enkele begrippen uit beeldverwerking}

We zullen in het vervolg van deze scriptie ook gebruik maken van enkele
begrippen en technieken uit beeldverwerking. In dit hoofdstuk geven we 
een overzicht van die begrippen en beschrijven we kort de verwante
technieken.

\section{Modellering van kleuren en beelden}

\subsection{kleurmodellen}

Een \emph{kleurmodel} is een abstract mathematisch model dat beschrijft hoe kleuren gerepresenteerd 
kunnen worden als $n$-tallen uit $\mathbb{R}^n$. RGB is het meest gebruikte kleurmodel. In dat model is elke kleur
een gewogen som van drie hoofdkleuren: rood, groen en blauw. De gewichten van die som
worden gebruikt als componenten van het drietal dat de kleur voorstelt. 
%Naast RGB zijn ook
%HSV en L*a*b* populaire kleurmodellen \cite{philips:beeldverwerking}.

De kleuren die op basis van een bepaald model kunnen voorgesteld worden, vormen een \emph{kleurruimte}. 
In het geval van RGB is dat een driedimensionale ruimte. De kleuren in die ruimte zijn afhankelijk
van de manier waarop men ``rood'', ``groen'' en ``blauw'' definieert. Veelgebruikte kleurruimtes 
op basis van het RGB-model zijn sRGB en Adobe RGB.
Hoewel het strikt gezien niet correct is, wordt de term ``kleurruimte'' vaak ook voor het
kleurmodel gebruikt. Men heeft het dus dikwijls over \emph{de} RGB kleurruimte, terwijl er eigenlijk meerdere
kleurruimtes bestaan die gebaseerd zijn op RGB. 

Ook in het HSV kleurmodel \cite{tkalcic:colour_spaces} wordt elke kleur voorgesteld 
door een drietal. Men noemt de componenten van zo'n drietal respectievelijk 
``hue'', ``saturation'' en ``value''. De eerste van die componenten 
correspondeert met de kleurtint, terwijl de overige componenten de saturatie
en de helderheid aangeven. Men kan een kleur $(r,g,b) \in [0,1]^3$ uit het RGB 
model als volgt omzetten naar een kleur $(h,s,v) \in [0,1]^3$ in het HSV-model: 
$$
\begin{array}{rcll}
h' & = & 60 \cdot \frac{g - b}{\max \{r,g,b\} - \min \{r,g,b\}}\quad & \textrm{ 
als } \max \{r,g,b\} = r \\[4pt] & = & 60 \cdot \frac{b - r}{\max \{r,g,b\} - 
\min \{r,g,b\}} + 120\quad & \textrm{ als } \max \{r,g,b\} = g \\[4pt] & = & 60 
\cdot \frac{r - g}{\max \{r,g,b\} - \min \{r,g,b\}} + 240\quad & \textrm{ als } 
\max \{r,g,b\} = b \\[6pt] h & = & \frac{h' \bmod 360}{360} & \textrm{ als } h' 
\bmod 360 \geq 0 \\[4pt] & = & \frac{360 - (h' \bmod 360)}{360} & \textrm{ als 
} h' \bmod 360 < 0 \\[6pt] s & = & \frac{\max \{r,g,b\} - \min \{r,g,b\}}{\max 
\{r,g,b\}} & \\[6pt] v & = & \max \{r,g,b\}
\end{array}
$$ Als $s=0$ dan is $h$ niet gedefinieerd. Dat is logisch vermits we dan te 
maken hebben met een grijswaarde. Indien $v=0$ dan zijn $s$ en $h$ niet gedefinieerd. We 
hebben het in dat geval dan ook over puur zwart, waarvoor er geen kleurtint of 
saturatie kan gespecificeerd worden. In de praktische implementatie geven we 
niet gedefinieerde componenten de waarde $0$.

Een kleur in het Irb-model bestaat uit een intensiteit en een genormaliseerde $r$ en $b$ 
\cite{ohta:color_info_for_region_segm}: $$ i = \frac{r+g+b}{3} \qquad r' = 
\frac{r}{r+g+b} \qquad b' = \frac{b}{r+g+b} $$ met $r,g,b \in [0,1]$ 
de co\"ordinaten van de beschouwde kleur in het RGB-model. 

Het I1I2I3-model, dat werd voorgesteld door Ohta 
\cite{ohta:color_info_for_region_segm}, is een \emph{opponent color space}. 
We zetten een kleur $(r,g,b) \in 
[0,1]^3$ uit het RGB-model als volgt om: $$
%\begin{array}{rcl}
i_1 = \frac{r+g+b}{3} \qquad i_2 = \frac{r-b}{2} \qquad i_3 = \frac{2 \cdot g - 
r - b}{4}
%\end{array}
$$ De eerste component is achromatisch (\emph{white-black}), terwijl de overige 
twee chromatisch zijn (\emph{red-green} en \emph{yellow-blue}).
We gebruiken hier, zoals in \cite{wang:cbir_using_daubechies_wavelets}, een 
genormaliseerde vorm van dit model: $$
%\begin{array}{rcl}
c_1 = \frac{r+g+b}{3} \qquad c_2 = \frac{r + (1 - b)}{2} \qquad c_3 = \frac{r + 
2 \cdot (1 - g) + b}{4}
%\end{array}
$$ 

In het XYZ-model wordt een kleur voorgesteld door de volgende drie componenten: 
$$
\begin{array}{rcl}
x & = & 0.4124 \cdot r + 0.3576 \cdot g + 0.1805 \cdot b \\
y & = & 0.2126 \cdot r + 0.7152 \cdot g + 0.0722 \cdot b \\
z & = & 0.0193 \cdot r + 0.1192 \cdot g + 0.9505 \cdot b
\end{array}
$$ met $(r,g,b) \in [0,1]^3$ de vector van de kleur in het RGB-model. Hierbij 
is $y$ evenredig met de luminantie van de kleur in kwestie. 

De eerste component van een kleur in het Yxy-model is gelijk aan de $y$ 
component van die kleur in het XYZ-model. De overige twee componenten berekenen 
we als volgt: $$ x' = \frac{x}{x+y+z} \qquad y' = \frac{y}{x+y+z} $$ 

Een kleurruimte op basis van het L*a*b*-model is \emph{perceptueel uniform} 
\cite{sharma:digital_color_imaging}. Dat wil zeggen dat, in een dergelijke 
ruimte, de Euclidische afstand een goede maat is voor het waargenomen verschil 
tussen twee kleuren. De co\"ordinaten $(l,a,b)$ van een kleur in het 
L*a*b*-model kunnen als volgt benaderd worden 
\cite{debaets:similariteitsmaten_voor_kleurbeelden, philips:beeldverwerking}: $$
\begin{array}{rcl}
l & = & 116 \cdot f(\frac{100 \cdot y}{255 \cdot y_0}) - 16 \\[5pt] a & = & 500 \cdot 
\left(f(\frac{100 \cdot x}{255 \cdot x_0}) - f(\frac{100 \cdot y}{255 \cdot y_0})\right) \\[5pt] b & = & 200 \cdot 
\left(f(\frac{100 \cdot y}{255 \cdot y_0}) - f(\frac{100 \cdot z}{255 \cdot z_0})\right)
\end{array}
$$ met $$
\begin{array}{rcll}
f(t) & = & t^\frac{1}{3} & \textrm{als } t > 0.008856 \\ & = & 7.787 \cdot t + 
\frac{16}{116} & \textrm{anders}
\end{array}
$$ voor elke re\"ele $t$. Hierbij zijn $x_0$, $y_0$ en $z_0$ de
XYZ-co\"ordinaten van een wit referentiepunt. We gebruiken hier 
$255 \cdot (x_0,y_0,z_0)=(95.05,100,108.9)$.
We kunnen dan als volgt normaliseren: $$ l' = \frac{l}{100} \qquad a' = \frac{120 + 
\max\{\min\{a,120\},-120\}}{240} \qquad b' = \frac{120 + \max\{\min\{b,120\},-120\}}{240} $$


\subsection{Kleurbeelden}

In een \emph{kleurbeeld} heeft elk beeldpunt een bepaalde kleur, die beschreven wordt met
behulp van een kleurmodel. Met andere woorden: de kleur van een beeldpunt wordt voorgesteld 
als een $n$-tal uit
$\mathbb{R}^n$. Zoals hierboven reeds een aantal keer werd ge\"illustreerd, kunnen dergelijke 
$n$-tallen genormaliseerd worden tot $n$-tallen uit $[0,1]^n$. We kunnen een $m$-dimensionaal 
kleurbeeld dus modelleren als een $\mathbb{R}^m - [0,1]^n$ afbeelding $b$. 
In de praktijk wordt een beeld echter voorgesteld als een rooster bestaande uit een
eindig aantal beeldpunten. Bijgevolg zal $b$ eerder een afbeelding van een deelverzameling van $\mathbb{N}^2$
naar $[0,1]^n$ zijn.

\section{Kleurkwantisatie}
\label{sectie:kleurkwantisatie}

Een kleurruimte bevat meestal een groot aantal 
kleuren. Zo gebruikt men voor een RGB-ruimte bijvoorbeeld typisch 8 bits per 
kleurcomponent. Dit geeft een totaal van 24 bits per kleur, zodat men dus 
$2^{24}=2^4 \cdot 2^{20}=16 \cdot 2^{20} \approx 16 \cdot 10^6$ kleuren kan 
voorstellen.

Om de complexiteit te beperken, kan het dus nodig zijn om het aantal kleuren te 
reduceren. Dat kan door middel van \emph{kleurkwantisatie}. Daarbij groeperen we de kleuren in 
zogenaamde \emph{bins}. We beschouwen dus geen aparte kleuren meer, maar wel 
verzamelingen van kleuren. De kleuren die in dezelfde bin zitten, worden dan 
vervangen door \'e\'en enkele kleur. Het \emph{kleurenpalet} is de verzameling 
van alle bin-kleuren. Bij eenvoudige kwantisatietechnieken wordt steeds 
hetzelfde palet gebruikt, onafhankelijk van de voor te stellen beelden. Meer 
geavanceerde technieken proberen een palet te gebruiken dat optimaal is voor de 
beschouwde beelden.

Wiskundig kunnen we kleurkwantisatie modelleren aan de hand van twee functies: 
$bin$ en $kleur$. De eerste functie associeert met elke kleur $c$, uit de 
kleurruimte $C$, het nummer van de corresponderende bin. Met behulp van de 
tweede functie kan men dan de kleur van die bin bepalen.

\subsection{Uniforme kwantisatie}

Bij \emph{uniforme kwantisatie} wordt elke kleurcomponent uniform verdeeld in een 
aantal intervallen. Als we zorgen dat $C=[0,1]^l$, $l > 0$, dan kunnen we de 
volgende definitie voor $bin$ gebruiken: $$
\begin{array}{lrcll}
bin: & C & \to & \{1,2,...,N\}\\[5pt] & (c_1,c_2,\ldots,c_l) & \mapsto & \left[ 
\displaystyle\sum_{i=1}^l \left( \prod_{j=1}^{i-1} N_j \right) \mathit{fl}(c_i, N_i) \right] 
+ 1, & \forall (c_1,c_2,\ldots,c_l) \in C
\end{array}
$$ met $N_j$ het aantal bins voor de $j$-de component, $N$ het totale aantal 
bins ($N=N_1 \cdot N_2 \cdot \ldots \cdot N_l$) en $$
\begin{array}{rcll}
\mathit{fl}(x,n) & = & \lfloor x \cdot n \rfloor & \textrm{als } x < 1 \\ & = & 
n - 1 & \textrm{als } x = 1
\end{array}
$$ voor alle $x \in [0,1]$ en $n \in \mathbb{N}$. De middelste waarden van de 
intervallen doen hierbij dienst als componenten van de kleur van een bin: $$
\begin{array}{lrcll}
kleur: & \{1,2,\ldots,N\} & \to & C \\ & n & \mapsto & (kleur_1(n), kleur_2(n), 
\dots, kleur_l(n)), & \forall n \in \{1,2,\ldots,N\}
\end{array}
$$ waarbij $$ kleur_i(n) = \frac{n}{\prod_{j=1}^i N_j} + \frac{1}{2 \cdot N_i} 
$$ voor alle $n \in \{1,2,\ldots,N\}$ en $i \in \{1,2,\ldots,l\}$.

Bij image retrieval is doorgaans vooral de kleurtint op zich belangrijk. Door
minder belang te hechten aan de luminantie kan de prestatie zelfs verbeteren,
vermits de similariteitsmaten dan bijvoorbeeld minder afhankelijk zijn van
belichtingsvariaties. 

Een mogelijke manier om uniform te kwantiseren is dus
het HSV-model gebruiken en meer belang hechten aan de eerste kleurcomponent. 
Dat doen we door $N_1=16$, $N_2=4$ en $N_3=4$ te kiezen.
%Bovendien is het discriminerend
%vermogen van dit histogram waarschijnlijk ook beter dan dat van een ongecomprimeerd kleurhistogram.
Die keuzes voor $N_1$, $N_2$ en $N_3$ stemmen overeen met de keuzes die men 
gemaakt heeft voor de \emph{scalable color descriptor} (SCD), die gedefinieerd 
wordt in de MPEG-7 standaard \cite{manjunath:color_and_texture_descriptors}.

Via analoge redeneringen voor de andere kleurmodellen, bekomen we de zes manieren 
uit tabel~\ref{tab:uniforme_kwantisatie} om uniform te kwantiseren met een vast kleurenpalet.
Elk van die manieren gebruikt $256$ bins.

\begin{table}[tbp]
\begin{center}
\begin{tabular}{|c|cccccc|}
\hline
 		& HSV & Irb & I1I2I3 & XYZ & Yxy & L*a*b* \\
\hline
$N_1$ 	& 16 & 4 & 4 & 8 & 4 & 4 \\
$N_2$	& 4  & 8 & 8 & 4 & 8 & 8 \\
$N_3$	& 4  & 8 & 8 & 8 & 8 & 8 \\
\hline
\end{tabular}
\caption{\label{tab:uniforme_kwantisatie}Zes manieren om uniform te kwantiseren met een vast kleurenpalet.}
\end{center}
\end{table}


\subsection{Niet-uniforme kwantisatie}

We hebben reeds vermeld dat bij image retrieval vooral de tinten van de kleuren belangrijk zijn. Daarom 
lijkt het op het eerste zicht geen slecht idee om kleuren met een zelfde kleurtint te groeperen in 
\'e\'en enkele bin. De eerste HSV-component bepaalt dan tot welke bin een kleur behoort. We passen
met andere woorden uniforme HSV-kwantisatie toe met $(N_1,N_2,N_3)=(256,1,1)$. Een probleem daarbij 
is echter dat beelden ook grijswaarden kunnen bevatten. Voor grijswaarden is de ``hue''-component immers 
niet gedefinieerd. De waarde 0 gebruiken voor niet-gedefinieerde kleurtinten lost dit probleem maar 
gedeeltelijk op, want dan komen alle grijswaarden in dezelfde bin terecht. Dat kan wel vermeden worden 
door bijvoorbeeld $(N_1,N_2,N_3)=(64,1,4)$ te kiezen, maar dan kunnen we maar 64 tinten meer onderscheiden.

Met behulp van niet-uniforme kwantisatie komen we tot een betere oplossing. Als we het HSV-model van naderbij
bekijken, dan merken we dat elke kleur getransformeerd kan worden in een grijstint door de saturatie
voldoende te verlagen. De waarde van de saturatie waarbij die transformatie plaatsvindt, hangt af
van de helderheid. We gaan daarom als volgt te werk. Als $s > 1 - 0.8 \cdot v$ dan benaderen we de kleur
met de kleurtint en kwantiseren we uniform met $(N_1,N_2,N_3)=(240,1,1)$. In het andere geval hebben we
te maken met een grijstint en kwantiseren we uniform met $(N_1,N_2,N_3)=(1,1,16)$. We combineren dus
twee uniforme kwantisaties tot \'e\'en niet-uniforme. Vermits $240 + 16 = 256$
slagen we er zo in om 240 kleurtinten te onderscheiden, zonder het aantal bins te verhogen. Omdat het
``Perceptually Smooth Color Transition Histogram'' uit \cite{sural:perceptually_smooth_histogram} 
gebaseerd is op deze kwantisatietechniek, spreken we van Smooth Color Transition (SCT) kwantisatie.  

Een andere niet-uniforme manier om te kwantiseren, is het mappen van de kleuren op elf 
\emph{focale kleuren} \cite{van_den_broek:human_color_categorization_for_cbir}: zwart, wit, rood, groen, geen, blauw, bruin, paars, roze, oranje en grijs.
Experimenten hebben aangetoond dat de mens geneigd is om enkel die kleurcategorie\"en te gebruiken bij 
het beredeneren, onthouden en waarnemen van kleuren. We kunnen voor elke kleur de juiste mapping bepalen
door op zoek te gaan naar de focale kleur die in het L*a*b*-model het dichtst bij die gegeven kleur ligt.

\emph{Neural Image Quantization} (NeuQuant) \cite{dekker:neuquant} en de \emph{Wu quantizer} 
\cite{wu:color_quantization_by_dynamic_programming_and_principal_analysis} zijn twee
populaire kwantisatietechnieken die, in tegenstelling tot de technieken die we 
tot nu toe bekeken hebben, een variabel kleurenpalet gebruiken. De eerste techniek is gebaseerd op
Kohonen neurale netwerken, terwijl de tweede gebruik maakt van dynamisch programmeren
en ``principal analysis''. NeuQuant geeft doorgaans iets betere resultaten, maar Wu's techniek is
sneller. Uit figuur~\ref{fig:rekentijden_quant} blijkt echter dat het verschil in rekentijd zich
enkel manifesteert als er naar een relatief groot aantal kleuren gekwantiseerd wordt. Voor een
klein aantal kleuren is NeuQuant zelfs iets sneller. Anderzijds is Wu's techniek wel meer
geschikt voor kwantisatie naar een zeer klein aantal kleuren.
 

\begin{figure}[tbp]
\begin{center}
\subfigure[]{
\begin{minipage}[t]{0.4\textwidth}
\vspace{0pt}
\centering
\includegraphics[height=3.8cm]{plots/uniform_quant_small_filled.eps}
\end{minipage}
\begin{minipage}[t]{0.4\textwidth}
\vspace{0pt}
\centering
\includegraphics[height=4.1cm]{plots/neuquant_vs_wu_small_filled.eps}
\end{minipage}
\label{fig:rekentijden_quant_small}
}
\subfigure[]{
\begin{minipage}[t]{0.4\textwidth}
\vspace{0pt}
\centering
\includegraphics[height=3.8cm]{plots/uniform_quant_big_filled.eps}
\end{minipage}
\begin{minipage}[t]{0.4\textwidth}
\vspace{0pt}
\centering
\includegraphics[height=4.1cm]{plots/neuquant_vs_wu_big_filled.eps}
\end{minipage}
\label{fig:rekentijden_quant_big}
}
\caption{\label{fig:rekentijden_quant}Vergelijking van de rekentijden bij kwantisatie van (a) een klein en (b) een groot beeld.}
\end{center}
\end{figure}

\subsection{Enkele voorbeelden}

We illustreren de bovenstaande technieken door ze toe te passen op de beelden uit 
figuur~\ref{fig:kwantistatie_originelen}. De eerste twee van die beelden hebben we geconstueerd
door in het HSV-model twee componenten te laten vari\"eren en \'e\'en component constant te houden.
Bij figuur~\ref{fig:hsv_constant_s} is de saturatie steeds gelijk aan $1$ en hebben we de kleurtint 
horizontaal en de helderheid vertikaal laten vari\"eren. Figuur~\ref{fig:hsv_constant_v} werd 
op een analoge manier geconstrueerd, maar in dit geval is de helderheid constant.
De beelden die we bekomen door toepassing van de zes manieren om uniform te kwantiseren,
worden weergegeven in figuur~\ref{fig:kwantistatie_uniform}. Figuur~\ref{fig:kwantistatie_niet-uniform}
toont de resultaten van de vier niet-uniforme kwantistatietechnieken.

\begin{figure}[tbp]
\begin{center}
\subfigure[]{
\includegraphics[height=2.8cm]{images/hsv_constant_s.eps}
\label{fig:hsv_constant_s}
}
\subfigure[]{
\includegraphics[height=2.8cm]{images/hsv_constant_v.eps}
\label{fig:hsv_constant_v}
}
\subfigure[]{
\includegraphics[height=2.8cm]{images/flowers.eps}
\label{fig:flowers}
}
\subfigure[]{
\includegraphics[height=2.8cm]{images/autumn.eps}
\label{fig:autumn}
}
\caption{\label{fig:kwantistatie_originelen}De beelden waarvan we de kleuren gaan kwantiseren op een aantal verschillende manieren.}
\end{center}
\end{figure}

De technieken met een variabel kleurenpalet doen het duidelijk beter dan diegene met een vast palet. 
Anderzijds zijn de technieken met een vast kleurenpalet wel een stuk sneller. Bovendien is het soms niet
gewenst dat voor alle beelden een andere verzameling van kleuren gebruikt wordt. In deze scriptie gebruiken
we kwantisatie trouwens louter om meer performante similariteitsmaten te constueren. De visuele kwaliteit
van de gekwantiseerde beelden is daarbij minder belangrijk. Similariteitsmaten op basis van kwantisatietechnieken 
waarbij de gekwantiseerde beelden een lagere kwaliteit hebben, presteren immers niet noodzakelijk
slechter dan maten die gebaseerd zijn op ``betere'' kwantisatietechnieken. Zo zouden we bijvoorbeeld 
betere kwantisatie kunnen bekomen door extra luminantiedata te behouden, maar daardoor zullen
de similariteitsmaten doorgaans minder goed werken vermits ze dan gevoeliger worden voor 
belichtingsvariaties.
%kunnen in bepaalde gevallen immers beter 
%presteren dan maten die gebaseerd zijn op ``betere'' kwantisatietechnieken.

Merk ook op dat uit figuur~\ref{fig:kwantistatie_klein_aantal_kleuren} inderdaad blijkt
dat NeuQuant minder geschikt is dan Wu's techniek voor kwantisatie naar een zeer klein
aantal kleuren. 

\begin{figure}[tbp]
\begin{center}
\subfigure[]{
\includegraphics[height=2.8cm]{images/uniform_hsv_constant_s.eps}
\includegraphics[height=2.8cm]{images/uniform_hsv_constant_v.eps}
\includegraphics[height=2.8cm]{images/uniform_hsv_flowers.eps}
\includegraphics[height=2.8cm]{images/uniform_hsv_autumn.eps}
\label{fig:uniform_hsv}
}
\subfigure[]{
\includegraphics[height=2.8cm]{images/uniform_irb_constant_s.eps}
\includegraphics[height=2.8cm]{images/uniform_irb_constant_v.eps}
\includegraphics[height=2.8cm]{images/uniform_irb_flowers.eps}
\includegraphics[height=2.8cm]{images/uniform_irb_autumn.eps}
\label{fig:uniform_irb}
}
\subfigure[]{
\includegraphics[height=2.8cm]{images/uniform_i1i2i3_constant_s.eps}
\includegraphics[height=2.8cm]{images/uniform_i1i2i3_constant_v.eps}
\includegraphics[height=2.8cm]{images/uniform_i1i2i3_flowers.eps}
\includegraphics[height=2.8cm]{images/uniform_i1i2i3_autumn.eps}
\label{fig:uniform_i1i2i3}
}
\subfigure[]{
\includegraphics[height=2.8cm]{images/uniform_xyz_constant_s.eps}
\includegraphics[height=2.8cm]{images/uniform_xyz_constant_v.eps}
\includegraphics[height=2.8cm]{images/uniform_xyz_flowers.eps}
\includegraphics[height=2.8cm]{images/uniform_xyz_autumn.eps}
\label{fig:uniform_xyz}
}
\subfigure[]{
\includegraphics[height=2.8cm]{images/uniform_yxy_constant_s.eps}
\includegraphics[height=2.8cm]{images/uniform_yxy_constant_v.eps}
\includegraphics[height=2.8cm]{images/uniform_yxy_flowers.eps}
\includegraphics[height=2.8cm]{images/uniform_yxy_autumn.eps}
\label{fig:uniform_yxy}
}
\subfigure[]{
\includegraphics[height=2.8cm]{images/uniform_lab_constant_s.eps}
\includegraphics[height=2.8cm]{images/uniform_lab_constant_v.eps}
\includegraphics[height=2.8cm]{images/uniform_lab_flowers.eps}
\includegraphics[height=2.8cm]{images/uniform_lab_autumn.eps}
\label{fig:uniform_lab}
}
\caption{\label{fig:kwantistatie_uniform}De resultaten na uniforme kwantisatie op basis van (a) HSV, (b) Irb, (c) I1I2I3, (d) XYZ, (e) Yxy en (f) L*a*b*.}
\end{center}
\end{figure}

\begin{figure}[tbp]
\begin{center}
\subfigure[]{
\includegraphics[height=2.8cm]{images/sct_constant_s.eps}
\includegraphics[height=2.8cm]{images/sct_constant_v.eps}
\includegraphics[height=2.8cm]{images/sct_flowers.eps}
\includegraphics[height=2.8cm]{images/sct_autumn.eps}
\label{fig:sct}
}
\subfigure[]{
\includegraphics[height=2.8cm]{images/focal_constant_s.eps}
\includegraphics[height=2.8cm]{images/focal_constant_v.eps}
\includegraphics[height=2.8cm]{images/focal_flowers.eps}
\includegraphics[height=2.8cm]{images/focal_autumn.eps}
\label{fig:focal}
}
\subfigure[]{
\includegraphics[height=2.8cm]{images/neuquant_constant_s.eps}
\includegraphics[height=2.8cm]{images/neuquant_constant_v.eps}
\includegraphics[height=2.8cm]{images/neuquant_flowers.eps}
\includegraphics[height=2.8cm]{images/neuquant_autumn.eps}
\label{fig:neuquant}
}
\subfigure[]{
\includegraphics[height=2.8cm]{images/wu_constant_s.eps}
\includegraphics[height=2.8cm]{images/wu_constant_v.eps}
\includegraphics[height=2.8cm]{images/wu_flowers.eps}
\includegraphics[height=2.8cm]{images/wu_autumn.eps}
\label{fig:wu}
}
\caption{\label{fig:kwantistatie_niet-uniform}De resultaten na niet-uniforme kwantisatie op basis van (a) SCT, (b) focale kleuren, (c) NeuQuant en (d) de Wu quantizer.}
\end{center}
\end{figure}

\begin{figure}[tbp]
\begin{center}
\subfigure[]{
\includegraphics[height=2.8cm]{images/neuquant_constant_s_8.eps}
\includegraphics[height=2.8cm]{images/neuquant_constant_v_8.eps}
\includegraphics[height=2.8cm]{images/neuquant_flowers_8.eps}
\includegraphics[height=2.8cm]{images/neuquant_autumn_8.eps}
\label{fig:neuquant_8}
}
\subfigure[]{
\includegraphics[height=2.8cm]{images/wu_constant_s_8.eps}
\includegraphics[height=2.8cm]{images/wu_constant_v_8.eps}
\includegraphics[height=2.8cm]{images/wu_flowers_8.eps}
\includegraphics[height=2.8cm]{images/wu_autumn_8.eps}
\label{fig:wu_8}
}
\caption{\label{fig:kwantistatie_klein_aantal_kleuren}De resultaten na kwantisatie naar slechts 8 kleuren op basis van (a) NeuQuant en (b) de Wu quantizer.}
\end{center}
\end{figure}

\section{Lineaire filters}

Bij het \emph{filteren} van een 2-dimensionaal beeld $b$ wordt de waarde van elk beeldpunt $(x,y) \in \mathbb{N}^2$ 
vervangen door een nieuwe waarde die afhangt van de originele waarde $b(x,y)$ van het beeldpunt en van de waarden 
van naburige beeldpunten. In het geval van een \emph{lineair filter} is die nieuwe waarde een lineaire combinatie 
van de oude waarden. Het beeld $b'$ dat we bekomen door $b$ lineair te filteren, wordt gegeven door
de volgende correlatie \cite{philips:beeldverwerking}:
$$
b'(x,y) = \sum_{(k,l) \in \Omega_m} b(x+k,y+l) \cdot m(k,l)
$$
met $m$ een $\mathbb{N}^2 - \mathbb{R}$ afbeelding en $\Omega_m = \{ (k,l) \in \mathbb{N}^2 \mid m(k,l) \ne 0 \}$. 
De functie $m$ wordt het \emph{filtermasker} genoemd.

Het \emph{3x3 binomiaalfilter} is een voorbeeld van een lineair filter. Voor het filtermasker 
$m_{bin}$ van dat filter geldt $m_{bin}(-1,-1)=m_{bin}(-1,1)=m_{bin}(1,-1)=m_{bin}(1,1)=1/16$, 
$m_{bin}(-1,0)=m_{bin}(0,-1)=m_{bin}(1,0)=m_{bin}(0,1)=1/8$, $m_{bin}(0,0)=1/4$ en 
$m_{bin}(k,l)=0$ voor de overige $(k,l)$. Als we veronderstellen dat $m_{bin}(k,l)=0$ voor de 
posities $(k,l)$ die niet weergegeven worden, dan kunnen we dat masker noteren als een matrix:
$$
\frac{1}{16}\left[ \begin{array}{ccc} 1 & 2 & 1\\ 2 & 4 & 2\\ 1 & 2 & 1 \end{array} \right]
= \frac{1}{16}\left[ \begin{array}{c} 1\\ 2\\ 1 \end{array} \right] \cdot 
\left[ \begin{array}{ccc} 1 & 2 & 1 \end{array} \right]
$$
Zoals ge\"illustreerd in figuur~\ref{fig:indische_ruizig_en_binom}, kan een binomiaalfilter gebruikt worden om ruis te onderdrukken.

\begin{figure}[tb]
\begin{center}
\subfigure[]{
\includegraphics[width=0.3\textwidth]{images/indische_ruizig.eps}
\label{fig:indische_ruizig}
}
\hspace{1cm}
\subfigure[]{
\includegraphics[width=0.3\textwidth]{images/indische_binom.eps}
\label{fig:indische_binom}
}
\caption{\label{fig:indische_ruizig_en_binom}Een ruizig beeld (a) en het resultaat na filteren met een binomiaalfilter (b).}
\end{center}
\end{figure}

\section{Eenvoudige randdetectie}

Beeldranden in een 2-dimensionaal beeld $b$ zijn plaatsen waar de luminantiecomponent $l(x,y)$ van $b(x,y)$ 
sterk varieert als functie van $x$ en/of $y$. Op die plaatsen zullen de partieel afgeleiden $D_1 l(x,y)$ en 
$D_2 l(x,y)$ dus groot zijn. Bijgevolg zal de \emph{gradi\"ent} $\nabla l$ daar ook groot zijn. We kunnen 
dus de volgende formule gebruiken om randen te detecteren in $b$: 
$$
|\nabla l(x,y)| / \sqrt{2} = \sqrt{(D_1 l(x,y))^2 + (D_2 l(x,y))^2} / \sqrt{2}
$$ 
met $l$ de $\mathbb{N}^2 - [0,1]$ functie die met elk beeldpunt $(x,y)$ de juiste luminantiecomponent associeert.

Doordat $l$ in de praktijk geen continue maar een discrete functie is, moeten we de partieel afgeleiden benaderen.
De Taylorreeksontwikkeling van een $\mathbb{R} - \mathbb{R}$ functie $f$ rond $x$ geeft
$$
f(x+h) = f(x) + h D f(x) + \frac{h^2}{2!} D^2 f(x) + \ldots
$$
en ook
$$
f(x-h) = f(x) - h D f(x) + \frac{h^2}{2!} D^2 f(x) + \ldots
$$
zodat
$$
\frac{f(x+h) - f(x-h)}{2h} \approx D f(x)
$$
Bijgevolg geldt $D_1 l(x,y) \approx (l(x+1,y) - l(x-1,y))/2$ en $D_2 l(x,y) \approx (l(x,y+1) - l(x,y-1))/2$. We
kunnen de eerste en tweede partieel afgeleiden dus berekenen met behulp van lineaire filters met filtermaskers:
$$
D_1 l(x,y)\textrm{: }\quad \frac{1}{2} \left[ \begin{array}{ccc} -1 & 0 & 1 \end{array} \right] \qquad \textrm{ en } 
\qquad D_2 l(x,y)\textrm{: }\quad \frac{1}{2} \left[ \begin{array}{c} -1 \\ 0 \\ 1 \end{array} \right]
$$
De \emph{Sobel-operator} is een variant hierop die minder ruisgevoelig is doordat er 
extra ruisonderdrukking voorzien wordt via een binomiaalfilter loodrecht op de richting 
van de parti\"ele afleiding:
$$
D_1 l(x,y)\textrm{: }\quad \frac{1}{8} \left[ \begin{array}{c} 1 \\ 2 \\ 1 \end{array} \right] \cdot \left[ \begin{array}{ccc} -1 & 0 & 1 \end{array} \right] \qquad \textrm{ en } 
\qquad D_2 l(x,y)\textrm{: }\quad \frac{1}{8} \left[ \begin{array}{c} -1 \\ 0 \\ 1 \end{array} \right] \cdot \left[ \begin{array}{ccc} 1 & 2 & 1 \end{array} \right]
$$
Uit figuur~\ref{fig:randdetectie} blijkt echter dat beide technieken geen spectaculair verschillende 
resultaten geven.

\begin{figure}[tbp]
\begin{center}
\subfigure[]{
\begin{tabular}{@{}c@{}}
\includegraphics[width=0.3\textwidth]{images/lena.eps}\\
\includegraphics[width=0.3\textwidth]{images/lena_met_ruis.eps}
\end{tabular}
\label{fig:lena}
}
%\hspace{0.5cm}
\subfigure[]{
\begin{tabular}{@{}c@{}}
\includegraphics[width=0.3\textwidth]{images/edges_lena.eps}\\
\includegraphics[width=0.3\textwidth]{images/edges_lena_met_ruis.eps}
\end{tabular}
\label{fig:lena_gradient}
}
%\hspace{0.5cm}
\subfigure[]{
\begin{tabular}{@{}c@{}}
\includegraphics[width=0.3\textwidth]{images/edges_sobel_lena.eps}\\
\includegraphics[width=0.3\textwidth]{images/edges_sobel_lena_met_ruis.eps}
\end{tabular}
\label{fig:lena_sobel}
}
\caption{\label{fig:randdetectie}Het origineel luminantiebeeld (a) en het resultaat na filteren met een gradi\"ent-filter (b) en met een Sobel-filter (c).}
\end{center}
\end{figure}


% \begin{figure}[tbp]
% \begin{center}
% \subfigure[]{
% \includegraphics[width=0.25\textwidth]{images/lena_met_ruis.eps}
% \label{fig:lena_met_ruis}
% }
% \hspace{0.5cm}
% \subfigure[]{
% \includegraphics[width=0.25\textwidth]{images/edges_lena_met_ruis.eps}
% \label{fig:lena_met_ruis_gradient}
% }
% \hspace{0.5cm}
% \subfigure[]{
% \includegraphics[width=0.25\textwidth]{images/edges_sobel_lena_met_ruis.eps}
% \label{fig:lena_met_ruis_sobel}
% }
% \caption{\label{fig:randdetectie_met_ruis}Het origineel ruizig luminantiebeeld (a) en het resultaat na filteren met een gradi\"ent-filter (b) en met een Sobel-filter (c).}
% \end{center}
% \end{figure}