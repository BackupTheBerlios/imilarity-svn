\chapter{Slotbeschouwingen}

In dit laatste hoofdstuk vatten we eerst de belangrijkste resultaten van dit eindwerk
nog even samen. Daarna vermelden we ook nog enkele interessante overblijvende
onderzoekmogelijkheden.

\section{Samenvatting van resultaten}

De basisidee achter deze scriptie is het identificeren van beelden
met vaagverzamelingen, zodanig dat vaagsimilariteitsmaten kunnen gebruikt worden
om de graduele gelijkenis tussen twee beelden na te gaan. 
We construeren dus similariteitsmaten voor beelden door een manier om beelden te
identificeren met vaagverzamelingen, te combineren met een vaagsimilariteitsmaat.
De eerste bijdrage van dit eindwerk is het herschrijven van de beschouwde
vaagsimilariteitsmaten naar een vorm die, onder bepaalde voorwaarden, een stuk
sneller kan berekend worden. 

We hebben twee types van similariteitsmaten voor beelden besproken.
De maten van het eerste type, waarbij het universum bestaat uit beeldpunten, 
noemen we pixelgebaseerde similariteitsmaten. Similariteitsmaten die 
beelden rechtstreeks interpreteren als vaagverzamelingen en vervolgens 
vaagsimilariteitsmaten gebruiken voor het vergelijken van die vaagverzamelingen, 
zijn pixelgebaseerd. Dergelijke maten kunnen echter niet gebruikt worden voor 
beelden met verschillende resoluties. Om dat probleem op te lossen hebben we 
similariteitsmaten op basis van intermediaire beelden beschouwd. Bovendien 
hebben we aangetoond dat ook beeldonderdelen een oplossing kunnen bieden voor 
het resolutieprobleem. 

Door het universum te laten bestaan uit kleuren, bekomen we maten van het tweede
type. Die maten noemen we kleurgebaseerde similariteitsmaten. Similariteitsmaten
die gebruik maken van histogrammen, zijn voorbeelden van kleurgebaseerde 
maten. Omdat vaagsimilariteitsmaten niet rechtstreeks kunnen
toegepast worden op traditionele histogrammen, hebben we pseudo-vage histogrammen
ge\"introduceerd. We hebben ook het vaaghistogram voorgesteld als alternatief.
Dat vaaghistogram heeft als voordeel dat het minder gevoelig is voor
problemen die optreden als gevolg van de kleurkwantisatie. Om ook bij de pseudo-vage
histogrammen de invloed van de kwantisatieproblemen te beperken, hebben we 
uitgebreide histogrammen gedefinieerd. 

Bij de pixelgebaseerde maten blijkt het componentsgewijs vergelijken van intermediaire
beelden goed te werken. In het bijzonder leiden de intermediaire beelden op basis van 
kleurmomenten tot verrassend goede resultaten. Bij de histogrammen presteert het 
Irb-histogram het best. In \ref{sectie:tests_banaan} hebben we echter een voorbeeld 
gezien waarbij het glad SCT-histogram betere resultaten oplevert. Verder hebben we
ook vastgesteld dat de vaagsimilariteitsmaat $M_{I_3}$ blijkbaar zeer 
geschikt is voor het vergelijken van histogrammen.

Voor de testcollectie uit figuur~\ref{fig:testcollectie} bekomen we de laagste 
GGGR-waarde als we $M_3$, $M_6$, $M_7$ of $M_{I_3}$ afzonderlijk toepassen op 
de componenten
van intermediaire beelden op basis van kleurmomenten, om vervolgens de
similariteiten van de verschillende componenten te aggregeren met behulp van
het minimum. Het verschil met de 
similariteitsmaat die gebaseerd is op het Irb-histogram en $M_{I_3}$ is echter 
zeer klein. Bovendien bevatten de kleurmomenten minder informatie dan het
Irb-histogram, zodat het Irb-histogram mogelijkerwijs beter zal presteren voor
meer realistische beelden. Daarom hebben we de similariteitsmaat op basis
van het Irb-histogram gekozen als standaard maat voor het prototype.

We hebben de geconstrueerde similariteitsmaten ge\"implementeerd in een
Java framework. Dat framework vormt de basis van twee applicaties. De ene
applicatie kan gebruikt worden om de performantie van de
verschillende similariteitsmaten na te gaan voor een bepaalde collectie
van beelden. Het andere programma is het eigenlijke prototype.

\section{Overblijvende onderzoeksmogelijkheden}

