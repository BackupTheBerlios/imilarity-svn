\chapter{Slotbeschouwingen}

In dit laatste hoofdstuk vatten we eerst de belangrijkste resultaten
nog even samen. Daarna vermelden we ook nog enkele interessante overblijvende
onderzoekmogelijkheden.

\section{Samenvatting van resultaten}

De basisidee achter deze scriptie is het identificeren van beelden
met vaagverzamelingen, zodanig dat vaagsimilariteitsmaten kunnen gebruikt worden
om de graduele gelijkenis tussen twee beelden na te gaan. 
We construeren dus similariteitsmaten voor beelden door een manier om beelden te
identificeren met vaagverzamelingen, te combineren met een vaagsimilariteitsmaat.
Een eerste bijdrage van dit eindwerk is het herschrijven van de beschouwde
vaagsimilariteitsmaten naar een vorm die, onder bepaalde voorwaarden, een stuk
sneller kan berekend worden. 

We hebben twee types van similariteitsmaten voor beelden besproken.
De maten van het eerste type, waarbij het universum bestaat uit beeldpunten, 
noemen we pixelgebaseerde similariteitsmaten. Similariteitsmaten die 
beelden rechtstreeks interpreteren als vaagverzamelingen en vervolgens 
vaagsimilariteitsmaten gebruiken voor het vergelijken van die vaagverzamelingen, 
zijn pixelgebaseerd. Dergelijke maten kunnen echter niet gebruikt worden voor 
beelden met verschillende resoluties. Om dat probleem op te lossen hebben we 
similariteitsmaten op basis van intermediaire beelden beschouwd. Bovendien 
hebben we aangetoond dat ook beeldonderdelen een oplossing kunnen bieden voor 
het resolutieprobleem. 

Door het universum te laten bestaan uit kleuren, bekomen we maten van het tweede
type. Die maten noemen we kleurgebaseerde similariteitsmaten. Similariteitsmaten
die gebruik maken van histogrammen, zijn voorbeelden van kleurgebaseerde 
maten. Omdat vaagsimilariteitsmaten niet rechtstreeks kunnen
toegepast worden op traditionele histogrammen, hebben we pseudo-vage histogrammen
ge\"introduceerd. We hebben ook het vaaghistogram voorgesteld als alternatief.
Dat vaaghistogram heeft als voordeel dat het minder gevoelig is voor
problemen die optreden als gevolg van de kleurkwantisatie. Om ook bij de pseudo-vage
histogrammen de invloed van die kwantisatieproblemen te beperken, hebben we 
uitgebreide histogrammen gedefinieerd. Die uitgebreide histogrammen laten bovendien
ook toe om spatiale informatie te incorporeren

Bij de pixelgebaseerde maten blijkt het componentsgewijs vergelijken van intermediaire
beelden goed te werken. In het bijzonder leiden de intermediaire beelden op basis van 
kleurmomenten tot verrassend goede resultaten. Bij de histogrammen presteert het 
Irb-histogram het best. In \ref{sectie:tests_banaan} hebben we echter een voorbeeld 
gezien waarbij het glad SCT-histogram betere resultaten oplevert. Verder hebben we
ook vastgesteld dat de vaagsimilariteitsmaat $M_{I_3}$ blijkbaar zeer 
geschikt is voor het vergelijken van histogrammen.

Voor de testcollectie uit figuur~\ref{fig:testcollectie} bekomen we de laagste 
GGGR-waarde als we $M_3$, $M_6$, $M_7$ of $M_{I_3}$ toepassen op 
de componenten
van intermediaire beelden op basis van kleurmomenten, om vervolgens de
similariteiten van de verschillende componenten te aggregeren met behulp van
het minimum. Het verschil met de 
similariteitsmaat die gebaseerd is op het Irb-histogram en $M_{I_3}$ is echter 
zeer klein. Bovendien bevatten de kleurmomenten minder informatie dan het
Irb-histogram, zodat het Irb-histogram mogelijkerwijs beter zal presteren voor
meer realistische beelden. Daarom hebben we de similariteitsmaat op basis
van het Irb-histogram gekozen als standaard maat voor het prototype.

We hebben de geconstrueerde similariteitsmaten ge\"implementeerd in een
Java framework. Dat framework vormt de basis van twee applicaties. De ene
applicatie kan gebruikt worden om de performantie van de
verschillende similariteitsmaten na te gaan voor een bepaalde collectie
van beelden. Het andere programma is het eigenlijke prototype.

\section{Overblijvende onderzoeksmogelijkheden}

In \ref{sectie:rekentijd} hebben we vermeld dat het bij echte CBIR mogelijk is
om berekeneningen op voorhand te doen. Zo zouden de histogrammen van de beelden
bijvoorbeeld mee opgeslagen kunnen worden in de databank. Bij echte CBIR kunnen
er daarom ook meer complexe similariteitsmtaten gebruikt worden. Door de beelden
op voorhand te segementeren, zou het bijvoorbeeld mogelijk zijn om bij het vergelijken
rekening te houden met de vorm van de objecten in de beelden. In het 
geval van het similariteitsgebaseerd rangschikken van de zoekresultaten kunnen we 
geen gebruik maken van de vorm, vermits de segementatie aanleiding zou geven tot 
similariteitsmaten met een veel te lange rekentijd. In de context van echte CBIR is 
er bijgevolg nog zeer veel ruimte voor het construeren van nieuwe vage 
similariteitsmaten voor beelden. Met de nodige creativiteit is het uiteraard ook 
mogelijk om nog bijkomende similariteitsmaten te ontdekken die wel bruikbaar 
zijn voor de praktische toepassing die we in dit eindwerk beschouwd hebben.

De aggregatieoperatoren vormen een tweede luik van overblijvend onderzoek. In deze
scriptie hebben we ons beperkt tot zeer eenvoudige operatoren. Bovendien hebben
we de mogelijkheden van aggregatie niet ten volle uitgebuit. Het
aggregeren van similariteiten kan onder meer nuttig zijn om: (1) meerdere voorbeelden 
te ondersteunen, (2) de similariteiten van verschillende componenten samen te voegen
en (3) similariteitsmaten te combineren.
Beschouw bijvoorbeeld een similariteitsmaat die beelden onderverdeelt in 
verschillende (vage) gebieden. Voor elk gebied kan er dan een andere similariteitsmaat
gebruikt worden. Het zou zelfs nuttig kunnen zijn om meerdere maten toe te passen
op hetzelfde gebied, bijvoorbeeld \'e\'en voor de kleur en een andere voor de vorm.
De similariteiten die teruggegeven worden door al die similariteitsmaten, moeten
uiteindelijk samengevoegd worden. Bovendien kunnen sommige maten
componentsgewijs te werk gaan en moet het ook mogelijk zijn om de similariteit met
een verzameling van voorbeelden te bepalen. De operatoren die gebruikt worden voor
de verschillende aggregaties, kunnen uiteraard een nefaste invloed hebben op
de performantie. Het zou bijgevolg zeer interessant zijn om voor elk van de 
aggregaties op zoek te gaan naar bestaande en/of nieuwe operatoren die 
zeer geschikt zijn voor dat type van aggregatie. Bij het combineren van 
similariteitsmaten moet uiteraard ook onderzocht worden welke combinaties
tot de beste resultaten leiden.

Verder zou het ook zinvol zijn om op zoek te gaan naar optimale waarden
voor de verschillende parameters van de similariteitsmaten. We denken bijvoorbeeld
aan het aantal bins bij de histogrammen. Met behulp van statistische methoden is het
mogelijk om een theoretisch optimaal aantal bins te bepalen. Mits enige aanpassing
zijn die methoden ook bruikbaar in een CBIR context 
\cite{viet_tran:ir_with_statistical_color_descriptors}. Mogelijk verder onderzoek bestaat
erin om na te gaan of dergelijke optima ook nuttig kunnen zijn voor vage 
similariteitsmaten voor beelden. 

We kunnen de basisidee achter deze scriptie ook meer algemeen formuleren: het 
identificeren van objecten -- in de algemene betekenis van het woord -- met 
vaagverzamelingen, zodanig dat vaagsimilariteitsmaten kunnen gebruikt worden
om de graduele gelijkenis tussen twee objecten na te gaan. Vaagverzamelingen en
vaagsimilariteitsmaten kunnen met andere woorden gezien worden als algemene
funderingen voor toepassingen waarbij gebruikt gemaakt wordt van similariteit.
In het bijzonder kunnen ze aangewend worden voor de constructie van similariteitsmaten
voor andere multimediabestanden, zoals audio en video.