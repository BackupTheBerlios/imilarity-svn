%het titelblad vergt hier en daar enkele manuele ingrepen i.v.m.
%spatiering en dergelijke
\begin{titlepage}
\renewcommand{\baselinestretch}{1.1}
\Large
\begin{center}
\mbox{}\\[0cm]%de volgende lijnen genereren het tempeltje
\unitlength 1mm

\epsfysize 4cm \epsfclipon\epsffile{ruglogo.eps}\\
%\begin{picture}(0,20)
%\centering
%\put(0,11){\makebox(0,0)[b]{\font\aula=aula34 {\aula a}}}
%\put(0,6){\makebox(0,0)[b]{\font\futura=futura scaled 1100
%                 {\futura UNIVERSITEIT}}}
%\put(0,2){\makebox(0,0)[b]{\font\futura=futura scaled 1100
%                 {\futura GENT}}}
%\put(0,0){\makebox(0,0)[b]{\rule{21mm}{1pt}}}
%\end{picture}\\
{\Large  
Faculteit Wetenschappen\\
Vakgroep \vakgroep\\
Voorzitter: \voorzitter
}\\\vfill
\parbox{14 cm}{
{\huge\bfseries
\begin{center}
\sf\titel
\end{center}
}
}\\\vfill
door \auteur\\[3.3cm]
%we schrijven ``.\"  i.p.v. ``.'' zodat LaTeX weet dat dit een
%afkorting is (gevolgd door kleine spatie) en niet het einde van een
%zin (gevolgd door lange spatie)
%opmerking: ofwel: prof.\ dr.\ I.\ Lemahieu 
%opmerking: ofwel: prof.\ dr.\ ir.\ W.\ Philips
Promotor: \promotor \\
%Co-promotor: \copromotor \\
Thesisbegeleider: \begeleider 
\\\vfill
%        In samenwerking met BARCO GRAPHICS\\
Afstudeerwerk ingediend tot het behalen van de graad van
%dit moet uiteraard worden aangepast!
\richting\\[1cm]
Academiejaar \jaar
\end{center}
\renewcommand{\baselinestretch}{1}
\end{titlepage}





%%% Local Variables: 
%%% mode: latex
%%% TeX-master: "total"
%%% End: 
