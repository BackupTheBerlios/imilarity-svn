\newpage
\thispagestyle{plain}

\begin{center}
{\sf\huge \titel }\\[3mm]
door\\
{\Large\auteur{}} \\
\end{center}
\noindent Afstudeerwerk ingediend tot het behalen van de graad van
\richting
\vspace{3mm}\\
Academiejaar \jaar
\vspace{3mm}\\
\noindent Universiteit Gent\\
Faculteit Wetenschappen\\
\vspace{3mm}\\
\noindent Promotor: \promotor\\
\noindent Scriptiebegeleiders: \begeleider\\
%\noindent Co-promotor: \copromotor\\
\vfill

%\noindent {\bf\Large Samenvatting}\\[1mm]
\section*{Samenvatting}
De exponenti\"ele groei van het internet geeft aanleiding tot een paradoxale
situatie: hoe meer informatie er beschikbaar komt, hoe moeilijker het wordt
om binnen een redelijke tijd accurate informatie te vinden. Zoekmachines zijn momenteel
de meest gebruikte online service, met miljoenen zoekopdrachten per dag. Naast het zoeken
in teksten, groeit de aandacht voor zogenaamde multimedia zoeksystemen. 

In het bijzonder maken de internetgebruikers steeds meer gebruik van zoekmachines die toelaten om 
gigantische collecties van afbeeldingen te doorzoeken. Deze zoekmachines zijn echter nagenoeg 
allemaal tekst-gebaseerd. Dit houdt in dat het zoeken enkel steunt op het vergelijken van opgegeven 
trefwoorden met tekstuele annotaties die toegekend worden aan elke afbeelding. Het hoeft geen
betoog dat het voor de gebruiker niet altijd gemakkelijk is om op deze manier een aanvaarbaar 
resultaat te vinden. Bovendien worden deze annotaties grotendeels automatisch gegenereerd, 
waardoor ze vaak weinig relevant zijn.

De hierboven genoemde problemen kunnen opgelost worden door het zoeken te baseren op de inhoud 
van de afbeeldingen. De bestaande inhoud-gebaseerde systemen kunnen echter nog altijd niet 
toegepast worden op afbeeldingencollecties van dergelijke grote omvang. In deze scriptie geven 
we daarom een alternatief voor deze meer geavanceerde zoeksystemen, waarbij de grootte van 
te doorzoeken collectie geen probleem vormt. 

Dit alternatief kan gezien worden als een uitbreiding van de tekst-gebaseerde zoeksystemen.
De zoekactie begint nog steeds met het specifi\"eren van \'e\'en of meerdere trefwoorden, waarop
het systeem antwoord met een lijst van relevante afbeeldingen. We zorgen er nu echter voor
dat de gebruiker een aantal voorbeeld-afbeeldingen kan kiezen uit deze lijst, waarna het
systeem de zoekresultaten rangschikt op een manier die de gelijkenis met de voorbeelden uitbuit.

Voor het modelleren van de graduele gelijkenis tussen afbeeldingen, maken we gebruik van 
vaagsimilariteitsmaten. Daarnaast zullen we ook nog beroep doen op aggregatieoperatoren,
zowel voor het combineren van verschillende similariteitsmaten als voor het samenvoegen
van de voorbeeld-afbeeldingen. 
%\vspace{5mm}

%\noindent {\bf\Large Trefwoorden}\\[1mm]
\section*{Trefwoorden}
content-based image retrieval, vaagverzamelingen, similariteitsmaten, aggregatieoperatoren

