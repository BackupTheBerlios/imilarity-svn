\newpage
\thispagestyle{plain}

\begin{center}
{\bf \titel }\\[3mm]
door\\
\auteur{} \\
\end{center}
\noindent Afstudeerwerk ingediend tot het behalen van de graad van
\richting
\vspace{3mm}\\
Academiejaar \jaar
\vspace{3mm}\\
\noindent Universiteit Gent\\
Faculteit Wetenschappen\\
\vspace{3mm}\\
\noindent Promotor: \promotor\\
%\noindent Co-promotor: \copromotor\\
\vfill

\noindent {\bf Samenvatting}\\[1mm]
De volgende tekst is een voorbeeld: In ziekenhuizen wordt tegenwoordig
zeer veel digitale data 
gegenereerd. Een groot deel daarvan is afkomstig van medische
beeldvormende modaliteiten zoals Magnetic Resonance Imaging,
Computerized Tomography, Positron Emission Tomography en Single Photon
Emission Computed Tomography. In het Universitair Ziekenhuis van Gent
staan ze in voor meerdere honderden Gbytes aan gegevens per jaar. In
deze thesis wordt nagegaan hoe de data actueel behandeld wordt en wat
de rol van beeldcompressie kan zijn om de opslag- en
transmissieproblemen te verlichten. Wegens de hoge kwaliteitseisen in
de medische sector ligt hierbij de nadruk op verliesloze
compressietechnieken.

Er wordt een vergelijking gemaakt van de performantie van
verschillende algoritmen, waarbij compressieverhouding en
verwerkingssnelheid de belangrijkste parameters zijn. Naast de
state-of-the-art verliesloze beeldcompressietechnieken worden ook
algemene data\-compressietechnieken in de vergelijking betrokken.
Voor MR- en CT-beelden levert de beeldcompressietechniek CALIC de
grootste compressie op, terwijl voor PET- en SPECT-beelden de
datacompressietechnieken STAT en BZIP de beste zijn. Wanneer de
verwer\-kingssnelheid belangrijker is dan de compressieverhouding,
wordt er het best geopteerd voor de datacompressietechnieken GZIP of
COMPRESS.

Verder wordt ook onderzocht hoe driedimensionale predictieve
technieken de redundantie in medische volumebeelden kunnen
uitbuiten. Er wordt aangetoond dat lineaire predictie daar niet in
slaagt. Door middel van context-modellering wordt echter wel een
toename van de compressie bekomen.


\vspace{5mm}

\noindent {\bf Trefwoorden}\\[1mm]
verliesloze beeldcompressie, medische beeldensets

