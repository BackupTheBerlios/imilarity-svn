\newpage
\thispagestyle{plain}

\begin{center}
{\sf\huge \titel }\\[3mm]
door\\
{\Large\auteur{}} \\
\end{center}
\noindent Afstudeerwerk ingediend tot het behalen van de graad van
\richting
\vspace{3mm}\\
Academiejaar \jaar
\vspace{3mm}\\
\noindent Universiteit Gent\\
Faculteit Wetenschappen\\
\vspace{3mm}\\
\noindent Promotor: \promotor\\
\noindent Scriptiebegeleiders: \begeleider\\
%\noindent Co-promotor: \copromotor\\
\vfill

%\noindent {\bf\Large Samenvatting}\\[1mm]
\section*{Samenvatting}
De exponenti\"ele groei van het internet geeft aanleiding tot een paradoxale
situatie: hoe meer informatie beschikbaar komt, hoe moeilijker het wordt
om binnen een redelijke tijd accurate informatie te vinden. Zoekmachines zijn momenteel
de meest gebruikte online service, met miljoenen zoekopdrachten per dag. Naast het zoeken
in teksten, groeit de aandacht voor zogenaamde multimedia zoeksystemen. 
In het bijzonder maken internetgebruikers steeds meer gebruik van zoekmachines die toelaten om 
gigantische collecties van beelden te doorzoeken. Die zoekmachines zijn echter nagenoeg 
allemaal tekstgebaseerd. Dat houdt in dat het zoeken enkel steunt op het vergelijken van opgegeven 
trefwoorden met tekstuele annotaties die toegekend worden aan elk beeld. Het hoeft geen
betoog dat het voor de gebruiker niet altijd gemakkelijk is om op die manier een aanvaardbaar 
resultaat te vinden. Bovendien worden de annotaties grotendeels automatisch gegenereerd, 
waardoor ze vaak weinig relevant zijn.

Door het zoeken te baseren op de inhoud van de beelden, kunnen de hierboven genoemde problemen 
opgelost worden. De bestaande inhoudgebaseerde zoeksystemen zijn echter nog altijd niet 
in staat beeldencollecties van zeer grote omvang te doorzoeken. In deze scriptie geven 
we daarom een alternatief voor die inhoudgebaseerde systemen, waarbij de grootte van 
de te doorzoeken collectie geen probleem vormt. 
Dat alternatief kan gezien worden als een uitbreiding van de tekstgebaseerde zoeksystemen.
De zoekactie begint nog steeds met het specificeren van \'e\'en of meerdere trefwoorden, waarop
het systeem antwoordt met een lijst van relevante beelden. We zorgen er nu echter voor
dat de gebruiker een aantal voorbeelden kan kiezen uit die lijst, waarna het
systeem de zoekresultaten rangschikt op een manier die de gelijkenis met de voorbeelden uitbuit.

Het uiteindelijke doel van dit afstudeerwerk is de bouw van een prototype dat
het similariteitsgebaseerd rangschikken van de zoekresultaten implementeert.
Voor het modelleren van de graduele gelijkenis tussen beelden, maken we daarbij gebruik van 
vaagsimilariteitsmaten. Om meerdere voorbeelden te kunnen ondersteunen, doen we 
daarnaast ook nog beroep op aggregatieoperatoren.
%Daarnaast zullen we ook nog beroep doen op aggregatieoperatoren,
%zowel voor het combineren van verschillende similariteitsmaten als voor het ondersteunen van
%meerdere voorbeelden. 
%\vspace{5mm}

%\noindent {\bf\Large Trefwoorden}\\[1mm]
\section*{Trefwoorden}
similariteitsmaten voor kleurbeelden, vaagverzamelingen, content-based image retrieval, 
beeld\-analyse

