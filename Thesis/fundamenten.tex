\chapter{Wiskundige fundamenten}

In dit hoofdstuk introduceren we eerst enkele basisbegrippen uit de vaagverzamelingenleer. 
Daarna volgt een overzicht van de vaagsimilariteitsmaten en aggregatieoperatoren waarvan
we later gebruik zullen maken. We hebben dit deel van deze scriptie bewust zo 
bondig mogelijk gehouden. Voor meer informatie verwijzen
we naar \cite{kerre:vaagmodellen}, \cite{vanderweken:similariteitsmaten} en 
\cite{victor:aggregatieoperatoren}.


\section{Basisbegrippen uit de vaagverzamelingenleer}

\subsection{Posets en tralies}
\label{sectie:posets_en_tralies}

In tegenstelling tot de meeste andere gebieden van de wiskunde, wordt in de theorie der 
vaagverzamelingen het begrip ``orde'' ten volle ge\"exploreerd \cite{kerre:vaagmodellen}. Daarom beginnen we deze 
sectie met het defini\"eren van twee ordestructuren: \emph{partieel geordende verzameling} 
(\emph{poset}) en \emph{tralie}. We maken bovendien van de gelegenheid gebruik om enkele 
belangrijke begrippen in verband met posets te introduceren. 

\begin{definitie}
Een partieel geordende verzameling (poset) is een koppel $(P,\le)$ bestaande uit een niet-ledige
verzameling $P$ en een binaire relatie $\le$ over $P$ die reflexief, antisymmetrisch en 
transitief is:
\begin{itemize}
  \item[(P.1)] $(\forall x \in P)(x \le x)$
  \item[(P.2)] $(\forall (x,y) \in P^2)(x \le y \land y \le x \Rightarrow x = y)$
  \item[(P.3)] $(\forall (x,y,z) \in P^3)(x \le y \land y \le z \Rightarrow x \le z)$
\end{itemize}
De relatie $\le$ wordt een (parti\"ele) orderelatie over $P$ genoemd.
\end{definitie} 
\begin{definitie}
Zij $(P,\le)$ een poset en $A \subseteq P$. Een element $b$ van $P$ is een maximaal element
van $A$ als en slechts als $b \in A$ en $(\forall x \in A)(b \le x \Rightarrow x = b)$. 
\end{definitie}
\begin{definitie}
Zij $(P,\le)$ een poset en $A \subseteq P$. Een element $b$ van $P$ is een minimaal element
van $A$ als en slechts als $b \in A$ en $(\forall x \in A)(x \le b \Rightarrow x = b)$. 
\end{definitie}
\begin{definitie}
Zij $(P,\le)$ een poset en $A \subseteq P$. Een element $b$ van $P$ is een bovengrens voor $A$
als en slechts als $(\forall a \in A)(a \le b)$. Als bovendien geldt $b \in A$, dan noemen
we $b$ het grootste element van $A$. De kleinste bovengrens voor $A$ is het
supremum voor $A$.
\end{definitie}
\begin{definitie}
Zij $(P,\le)$ een poset en $A \subseteq P$. Een element $b$ van $P$ is een ondergrens voor $A$
als en slechts als $(\forall a \in A)(b \le a)$. Als bovendien geldt $b \in A$, dan noemen
we $b$ het kleinste element van $A$. De grootste ondergrens voor $A$ is het
infimum voor $A$.
\end{definitie}
\begin{definitie}
Een poset $(L,\le)$ waarin elk doubleton een supremum en een infimum bezit, noemt men een tralie 
(lattice). Als elke niet-ledige deelverzameling van $L$
een supremum en een infimum bezit, dan is $(L, \le)$ een complete tralie.
\end{definitie}
% ALTERNATIEVE DEFINITIE VAN TRALIE
%% enkele nieuwe commando's die we verder in deze sectie nodig zullen hebben
%\newcommand{\dotand}{\ensuremath{\mathaccent\ldotp\land}}
%\newcommand{\dotor}{\ensuremath{\mathaccent\cdotp\lor}}
%\begin{definitie}
%Een algebra\"ische structuur $(L,\dotor,\dotand)$ bestaande uit een niet-ledige
%verzameling $L$ en twee binaire bewerkingen op $L$ wordt een tralie genoemd als en slechts als
%voor elke $a$, $b$ en $c$ uit $L$ geldt:
%\begin{itemize}
%  \item[(L.1)] $a \dotand a = a$ en $a \dotor a = a$
%  \item[(L.2)] $a \dotand b = b \dotand a$ en $a \dotor b = b \dotor a$
%  \item[(L.3)] $a \dotand (b \dotand c) = (a \dotand b) \dotand c$ en $a \dotor (b \dotor c) = (a \dotor b) \dotor c$
%  \item[(L.4)] $a \dotand (a \dotor b) = a$ en $a \dotor (a \dotand b) = a$
%\end{itemize}
%\end{definitie}

\subsection{Vaagverzamelingen}

Het \emph{universum} is de collectie van alle mogelijke elementen (bijvoorbeeld de
natuurlijke getallen). Een verzameling bevat bepaalde elementen uit dat universum (bijvoorbeeld de 
verzameling van de priemgetallen). 

In het geval van een \emph{scherpe (deel)verzameling} behoort elk 
element uit het universum wel of niet tot de verzameling. Andere mogelijkheden zijn er
niet. Een dergelijke verzameling kan bijgevolg 
gerepresenteerd worden door een \emph{karakteristieke afbeelding}, die elk element uit het 
universum afbeeldt op 0 of 1. Dat getal wordt de \emph{lidmaatschapsgraad} van het element 
in kwestie genoemd. De klasse van scherpe verzamelingen in een universum $X$ stellen we voor door 
$\mathcal{P}(X)$.
\begin{definitie}
Zij $X$ een universum. De karakteristieke afbeelding $\mu_A$ van een scherpe verzameling $A$ in $X$
wordt gedefinieerd als de $X - \{0,1\}$ afbeelding:
\begin{displaymath}
\begin{array}{lllll}
\mu_A: 	& X & \to 		& \{0,1\}	& \\
		& x & \mapsto 	& 1,		& \textrm{ als } x \in A \\
		& x & \mapsto 	& 0,		& \textrm{ als } x \notin A
\end{array}
\end{displaymath}
\end{definitie}

Bij een \emph{vaagverzameling} kunnen alle waarden tussen 0 en 1 als lidmaatschapsgraad 
voorkomen. De ``karakteristieke'' afbeelding is in dat geval dus een $X - [0,1]$ afbeelding:
\begin{definitie}
Zij $X$ een universum. Een vaagverzameling $A$ in $X$ wordt gekarakteriseerd door een $X - [0,1]$
afbeelding  $\mu_A$:
\begin{displaymath}
\begin{array}{lllll}
\mu_A: 	& X & \to 		& [0,1]	& \\
		& x & \mapsto 	& \mu_A(x),		& \forall x \in A
\end{array}
\end{displaymath}
\end{definitie}
\noindent
Een element $x \in X$ behoort dus tot de vaagverzameling $A$ met lidmaatschapsgraad $\mu_A(x)$.
Voor de eenvoud noteren we in het vervolg $\mu_A(x)$ steeds als $A(x)$. We maken
met andere woorden geen onderscheid tussen de vaagverzameling en de 
lidmaatschapsfunctie. Voor de klasse van vaagverzamelingen in een universum $X$ gebruiken we
de notatie $\mathcal{F}(X)$.

De \emph{drager} en de \emph{kern} van een vaagverzameling zijn twee belangrijke begrippen: 
\begin{definitie}
De drager van een vaagverzameling $A$ in $X$ wordt gedefinieerd als:
\begin{displaymath}
supp\ A = \{x \in X \mid A(x) > 0\} 
\end{displaymath}
\end{definitie}
\begin{definitie}
De kern van een vaagverzameling $A$ in $X$ wordt als volgt gedefinieerd:
\begin{displaymath}
ker\ A = \{x \in X \mid A(x) = 1\}
\end{displaymath}
\end{definitie}
\noindent
Ook het begrip \emph{cardinaliteit} speelt vaak een belangrijke rol. De cardinaliteit van een 
eindige scherpe verzameling wordt gegeven door het aantal elementen in die verzameling. 
Dat concept kan uitgebreid worden naar vaagverzamelingen door gebruik te maken van het begrip 
\emph{sigma count}:
\begin{definitie}
De sigma count van een vaagverzameling $A$ met eindige drager in een universum $X$ wordt
gedefinieerd door:
\begin{displaymath}
|A|=\sum_{x \in X} A(x)
\end{displaymath}
\end{definitie}

\subsection{Bewerkingen op vaagverzamelingen}
\label{sectie:bew_op_vaagverz}

We beginnen met het defini\"eren van de begrippen \emph{negator}, \emph{conjunctor} en 
\emph{disjunctor}. Die operatoren zijn uitbreidingen van de klassieke logische operatoren
$\lnot$ (negatie), $\land$ (conjunctie) en $\lor$ (disjunctie).
\begin{definitie}
Een negator $\mathcal{N}$ op $[0,1]$ is een dalende $[0,1] - [0,1]$ afbeelding die voldoet
aan de randvoorwaarden $\mathcal{N}(0)=1$ en $\mathcal{N}(1)=0$. 
\end{definitie}
\begin{definitie}
Een conjunctor $\mathcal{C}$ op $[0,1]$ is een stijgende $[0,1]^2 - [0,1]$ afbeelding die voldoet aan de
randvoorwaarden $\mathcal{C}(0,0)=\mathcal{C}(0,1)=\mathcal{C}(1,0)=0$ en $\mathcal{C}(1,1)=1$. 
Als een conjunctor voldoet aan 
$(\forall a \in [0,1])(\mathcal{C}(1,a)=\mathcal{C}(a,1)=a)$ dan is het een semi-norm.
Een driehoeksnorm (t-norm) is een commutatieve en associatieve semi-norm.
\end{definitie}
\begin{definitie}
Een disjunctor $\mathcal{D}$ op $[0,1]$ is een stijgende $[0,1]^2 - [0,1]$ afbeelding die voldoet
aan de randvoorwaarden $\mathcal{D}(1,0)=\mathcal{D}(0,1)=\mathcal{D}(1,1)=1$ en 
$\mathcal{D}(0,0)=0$. Als een disjunctor voldoet aan 
$(\forall a \in [0,1])(\mathcal{D}(0,a)=\mathcal{D}(a,0)=a)$ dan is het een semi-conorm.
Een driehoeksconorm (t-conorm) is een commutatieve en associatieve semi-conorm.
\end{definitie}

De meest gebruikte negator is de standaardnegator $N_s$. Het minimum $T_M$ en het algebra\"isch 
product $T_P$ zijn veelgebruikte t-normen. Bij de t-conormen zijn het maximum $S_M$ en de
probabilistische som $S_P$ dan weer populaire mogelijkheden. Die operatoren worden als 
volgt gedefinieerd:
%\begin{displaymath}
%\begin{array}{r@{\quad=\quad}l}
\begin{align*}
N_s(x) &= 1 - x \\
T_M(x,y) &= \min \{x,y\} \\
T_P(x,y) &= x \cdot y \\
S_M(x,y) &= \max \{x,y\} \\
S_P(x,y) &= x + y - x \cdot y
\end{align*}
%\end{array}
%\end{displaymath}
voor alle $(x,y)$ in $[0,1]^2$.

We kunnen de bovenstaande operatoren nu gebruiken om de klassieke verzameltechnische bewerkingen 
$co$ (complement), $\cap$ (doorsnede) en $\cup$ (unie) te
veralgemenen tot bewerkingen op vaagverzamelingen.
\begin{definitie}
Het $\mathcal{N}$-complement $co_\mathcal{N} A$ van een vaagverzameling $A$ in $X$ wordt gedefinieerd
door de volgende vaagverzameling in X:
\begin{displaymath}
(co_\mathcal{N} A)(x) = \mathcal{N}(A(x)),
\end{displaymath}
voor alle $x$ in $X$, met $\mathcal{N}$ een negator.
\end{definitie}
\begin{definitie}
De $\mathcal{C}$-doorsnede $A \cap_\mathcal{C} B$ van twee vaagverzamelingen $A$ en $B$ in $X$
wordt gedefinieerd door de volgende vaagverzameling in $X$:
\begin{displaymath}
(A \cap_\mathcal{C} B)(x) = \mathcal{C}(A(x),B(x)),
\end{displaymath}
voor alle $x$ in $X$, met $\mathcal{C}$ een conjunctor.
\end{definitie}
\begin{definitie}
De $\mathcal{D}$-unie $A \cup_\mathcal{D} B$ van twee vaagverzamelingen $A$ en $B$ in $X$ wordt gedefinieerd door de
volgende vaagverzameling in $X$:
\begin{displaymath}
(A \cup_\mathcal{D} B)(x) = \mathcal{D}(A(x),B(x)),
\end{displaymath}
voor alle $x$ in $X$, met $\mathcal{D}$ een disjunctor.
\end{definitie}

In het vervolg van deze scriptie zullen we steeds 
$(\mathcal{N},\mathcal{C},\mathcal{D})=(N_s,T_M,S_M)$ kiezen. We voeren daarom de volgende
verkorte notaties in:
\begin{align*}
A^c 			&= co_{N_s} A \\
A \cap B 		&= A \cap_{T_M} B \\
A \cup B		&= A \cup_{S_M} B \\
A \setminus B  	&= A \cap B^c \\
A \triangle B 	&= (A \setminus B) \cup (B \setminus A)
\end{align*}
waarbij $A$ en $B$ vaagverzamelingen in een zelfde universum zijn.


\subsection{L-vaagverzamelingen}

Het begrip vaagverzameling kan op zijn beurt nog eens uitgebreid worden tot 
\emph{L-vaagverzameling}. Die uitbreiding is gebaseerd op het concept \emph{tralie}
uit sectie~\ref{sectie:posets_en_tralies}.
\begin{definitie}
Zij $X$ een universum en $(L,\le)$ een complete tralie. Een L-vaagverzameling $A$ in $X$ wordt
gekarakteriseerd door een $X - L$ afbeelding $\mu_A$:
\begin{displaymath}
\begin{array}{lllll}
\mu_A: 	& X & \to 		& L	& \\
		& x & \mapsto 	& \mu_A(x),		& \forall x \in A
\end{array}
\end{displaymath}
\end{definitie}
\noindent
De klasse van L-vaagverzamelingen in een universum $X$ stellen we voor door 
$\mathcal{F}_L(X)$. Merk op dat die klasse zich herleidt tot $\mathcal{F}(X)$ voor $L = [0,1]$.

\subsection{Bewerkingen op L-vaagverzamelingen}

De begrippen \emph{negator}, \emph{conjunctor} en 
\emph{disjunctor} kunnen uitgebreid worden naar L-vaag\-ver\-za\-me\-ling\-en. Dat heeft als 
gevolg dat de veralgemeende verzameltechnische bewerkingen $co$, $\cap$ en 
$\cup$ uit \ref{sectie:bew_op_vaagverz} eveneens toepasbaar zijn op L-vaagverzamelingen.
We geven hieronder de definities van die uitbreidingen.
Hierbij is $(L,\le)$ een complete tralie die $l$ als kleinste en $u$ als grootste element heeft.
\begin{definitie}
Een negator $\mathcal{N}$ op $L$ is een dalende $L - L$ afbeelding die voldoet
aan de randvoorwaarden $\mathcal{N}(l)=u$ en $\mathcal{N}(u)=l$. 
\end{definitie}
\begin{definitie}
Een conjunctor $\mathcal{C}$ op $L$ is een stijgende $L^2 - L$ afbeelding die voldoet aan de
randvoorwaarden $\mathcal{C}(l,l)=\mathcal{C}(l,u)=\mathcal{C}(u,l)=l$ en $\mathcal{C}(u,u)=u$. 
Als een conjunctor voldoet aan 
$(\forall a \in L)(\mathcal{C}(u,a)=\mathcal{C}(a,u)=a)$ dan is het een semi-norm.
Een driehoeksnorm (t-norm) is een commutatieve en associatieve semi-norm.
\end{definitie}
\begin{definitie}
Een disjunctor $\mathcal{D}$ op $L$ is een stijgende $L^2 - L$ afbeelding die voldoet
aan de randvoorwaarden $\mathcal{D}(u,l)=\mathcal{D}(l,u)=\mathcal{D}(u,u)=u$ en 
$\mathcal{D}(l,l)=l$. Als een disjunctor voldoet aan 
$(\forall a \in L)(\mathcal{D}(l,a)=\mathcal{D}(a,l)=a)$ dan is het een semi-conorm.
Een driehoeksconorm (t-conorm) is een commutatieve en associatieve semi-conorm.
\end{definitie}

\section{Vaagsimilariteitsmaten}
\label{sectie:vaagsimilariteitsmaten}

Het nagaan van de graduele gelijkenis tussen twee vaagverzamelingen -- of tussen twee objecten die 
identificeerbaar zijn met vaagverzamelingen -- kan aan de hand van \emph{vaagsimilariteitsmaten}. 
Een dergelijke similariteitsmaat voor het vergelijken van twee vaagverzamelingen in een 
universum $X$, is niets meer dan een vaagverzameling in $\mathcal{F}(X) \times \mathcal{F}(X)$. De 
lidmaatschapsgraad van een koppel $(A,B) \in (\mathcal{F}(X))^2$ nadert daarbij naar 1 naarmate 
de similariteit tussen de vaagverzamelingen $A$ en $B$ toeneemt.

In \cite{vanderweken:similariteitsmaten} worden er 32 vaagsimilariteitsmaten voorgesteld: $M_1$, $M_2$,
$M_3$, $M_4$, $M_5$, $M_{5c}$, $M_6$, $M_{6c}$, $M_7$, $M_{7c}$, $M_8$, $M_{8c}$, $M_9$, $M_{9c}$,
$M_{10}$, $M_{10c}$, $M_{11}$, $M_{11c}$, $M_{12}$, $M_{13}$, $M_{14a}$, $M_{14b}$, $M_{14c}$, 
$M_{16e}$, $M_{16h}$, $M_{17a}$, $M_{17b}$, $M_{I_3}$, $M_{I_{3c}}$, $M_{18c}$, $M_{20}$ en $M_{20c}$.
We gaan die echter niet allemaal gebruiken. Vooreerst beperken we ons voor
$$
M_1(A,B) = 1 - \left(\frac{1}{|X|} \sum_{x \in X} | A(x) - B(x)|^r\right)^\frac{1}{r} \textrm{ met } 
r \in \{1,2,3,\ldots\}
$$ tot de keuzes
$r=1,2,4$. De overeenkomstige similariteitsmaten noemen we $M_{1a}$, $M_{1b}$ en $M_{1c}$. Vermits
$M_{1a}=M_{18c}$ hoeven we $M_{18c}$ dan niet meer te beschouwen. 
Bovendien zullen we ons enkel
toespitsen op de maten die rechtstreeks uitbreidbaar zijn naar L-vaagverzamelingen met $L=[0,1]^3$. 
Dit houdt in dat we enkel de maten beschouwen die op evidente wijze kunnen veralgemeend worden naar
vaagverzamelingen in $\mathcal{F}_L(X) \times \mathcal{F}_L(X)$. De details van die
uitbreiding behandelen we in \ref{sectie:pixelgeb_kleurbeelden}. 

Zo komen we uiteindelijk tot
de similariteitsmaten die worden weergegeven in tabel~\ref{tab:similatiteitsmaten}. 
Alle vaagsimilariteitsmaten uit die tabel zijn reflexief en symmetrisch. Dat wil zeggen dat 
voor elke maat $M$ geldt: $M(A,A)=1$ en $M(A,B)=M(B,A)$. 

\begin{table}[tbp]
\begin{center}
\begin{tabular}{|l|}
\hline
\\[1pt]
$
\begin{array}{r@{\ }c@{\ }l}
\displaystyle M_{1a}(A,B) & = & \displaystyle 1-\frac{1}{|X|}\sum_{x \in X} | A(x) - B(x) | \\
\displaystyle M_{1b}(A,B) & = & \displaystyle 1-\left(\frac{1}{|X|}\sum_{x \in X} | A(x) - B(x) |^2\right)^\frac{1}{2} \\
\displaystyle M_{1c}(A,B) & = & \displaystyle 1-\left(\frac{1}{|X|}\sum_{x \in X} | A(x) - B(x) |^4\right)^\frac{1}{4} \\
\end{array}
\begin{array}{r@{\ }c@{\ }l}
\displaystyle M_2(A,B) & = & \displaystyle 1 - \max_{x \in X} | A(x) - B(x) | \\[10pt]
\displaystyle M_3(A,B) & = & \displaystyle 1 - \frac{\displaystyle \sum_{x \in X} | A(x) - B(x) |}{\displaystyle |A| + |B|} \\
\end{array}
$
\\
\\[1pt]
\hline
\\[1pt]
$
\begin{array}{r@{\ }c@{\ }l}
\displaystyle M_5(A,B) & = & \displaystyle \frac{\min \{|A|,|B|\}}{\max \{|A|,|B|\}} \\[10pt]
\displaystyle M_{5c}(A,B) & = & \displaystyle \frac{\min \{|A^c|,|B^c|\}}{\max \{|A^c|,|B^c|\}} \\[10pt]
\displaystyle M_6(A,B) & = & \displaystyle \frac{|A \cap B|}{|A \cup B|} \\[10pt]
\displaystyle M_{6c}(A,B) & = & \displaystyle \frac{|A^c \cap B^c|}{|A^c \cup B^c|} \\[10pt]
\displaystyle M_7(A,B) & = & \displaystyle \frac{|A \cap B|}{\max \{|A|,|B|\}} \\[10pt]
\displaystyle M_{7c}(A,B) & = & \displaystyle \frac{|A^c \cap B^c|}{\max \{|A^c|,|B^c|\}} \\[10pt]
\displaystyle M_8(A,B) & = & \displaystyle \frac{|A \cap B|}{\min \{|A|,|B|\}} \\[10pt]
\displaystyle M_{8c}(A,B) & = & \displaystyle \frac{|A^c \cap B^c|}{\min \{|A^c|,|B^c|\}} \\[10pt]
\end{array}
\begin{array}{@{\qquad}r@{\ }c@{\ }l}
\displaystyle M_9(A,B) & = & \displaystyle \frac{\min \{|A|,|B|\}}{|A \cup B|} \\[10pt]
\displaystyle M_{9c}(A,B) & = & \displaystyle \frac{\min \{|A^c|,|B^c|\}}{|A^c \cup B^c|} \\[10pt]
\displaystyle M_{10}(A,B) & = & \displaystyle \frac{\max \{|A|,|B|\}}{|A \cup B|} \\[10pt]
\displaystyle M_{10c}(A,B) & = & \displaystyle \frac{\max \{|A^c|,|B^c|\}}{|A^c \cup B^c|} \\[10pt]
\displaystyle M_{11}(A,B) & = & \displaystyle \frac{\min \{|A \setminus B|,|B \setminus A|\}}{\max \{|A \setminus B|,|B \setminus A|\}} \\[10pt]
\displaystyle M_{11c}(A,B) & = & \displaystyle \frac{\min \{|(A \setminus B)^c|,|(B \setminus A)^c|\}}{\max \{|(A \setminus B)^c|,|(B \setminus A)^c|\}} \\[10pt]
\displaystyle M_{12}(A,B) & = & \displaystyle \frac{|(A \triangle B)^c|}{\max \{|(A \setminus B)^c|,|(B \setminus A)^c|\}} \\[10pt]
\displaystyle M_{13}(A,B) & = & \displaystyle \frac{|(A \triangle B)^c|}{\min \{|(A \setminus B)^c|,|(B \setminus A)^c|\}}
\end{array}
$
\\
\\[1pt]
\hline
\\[1pt]
$
\begin{array}{r@{\ }c@{\ }l}
\displaystyle M_{I_3}(A,B) & = & \displaystyle \frac{|(A \cap B) \cap (A^c \cap B^c)|}{|(A \cup B) \cap (A^c \cup B^c)|}
\end{array}
\begin{array}{@{\qquad}r@{\ }c@{\ }l}
\displaystyle M_{I_{3c}}(A,B) & = & \displaystyle \frac{|(A^c \cap B^c) \cup (A \cap B)|}{|(A^c \cup B^c) \cup (A \cup B)|}
\end{array}
$
\\
\\[1pt]
\hline
\end{tabular}
\caption{\label{tab:similatiteitsmaten}De vaagsimilariteitsmaten die we gaan gebruiken.}
\end{center}
\end{table}



\section{Aggregatieoperatoren} 