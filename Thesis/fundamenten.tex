\chapter{Wiskundige fundamenten}

In dit hoofdstuk introduceren we eerst enkele basisbegrippen uit de vaagverzamelingenleer. 
Daarna geven ook nog een overzicht van de similariteitsmaten en aggregatieoperatoren waarvan
we verder in deze scriptie gebruik zullen maken.


\section{Basisbegrippen uit de vaagverzamelingenleer}

\subsection{Vaagverzamelingen}

De collectie van alle mogelijke elementen noemen we het \emph{universum} (bijvoorbeeld de
natuurlijke getallen). Een verzameling bevat bepaalde elementen uit dit universum (bijvoorbeeld de 
verzameling van de priemgetallen). 

In het geval van een \emph{scherpe verzameling}, behoort elk 
element uit het universum wel of niet tot de verzameling. Andere mogelijkheden zijn er
niet. Een dergelijke verzameling kan bijgevolg 
gerepresenteerd worden door een \emph{karakteristieke afbeelding}, die elk element uit het 
universum afbeeldt op 0 of 1. Dit getal noemen we de \emph{lidmaatschapsgraad} van het element 
in kwestie. De klasse van scherpe verzamelingen in een universum $X$ stellen we voor door 
$\mathcal{P}(X)$.
\begin{definitie}
Zij $X$ een universum. De karakteristieke afbeelding $\mu_A$ van een scherpe verzameling $A$ in $X$
wordt gedefinieerd als de $X - \{0,1\}$ afbeelding:
$$
\begin{array}{lllll}
\mu_A: 	& X & \to 		& \{0,1\}	& \\
		& x & \mapsto 	& 1,		& \textrm{ als } x \in A \\
		& x & \mapsto 	& 0,		& \textrm{ als } x \notin A
\end{array}
$$
\end{definitie}

Bij een \emph{vaagverzameling} kunnen alle waarden tussen 0 en 1 als lidmaatschapsgraad 
voorkomen. De karakteristiek afbeelding is in dit geval dus een $X - [0,1]$ afbeelding:
\begin{definitie}
Zij $X$ een universum. Een vaagverzameling $A$ in $X$ wordt gekarakteriseerd door een $X - [0,1]$
afbeelding  $\mu_A$:
$$
\begin{array}{lllll}
\mu_A: 	& X & \to 		& [0,1]	& \\
		& x & \mapsto 	& \mu_A(x),		& \forall x \in A
\end{array}
$$
\end{definitie}
\noindent
Een element $x \in X$ behoort dus tot de vaagverzameling $A$ met lidmaatschapsgraad $\mu_A(x)$.
Voor de eenvoud zullen we in het vervolg $\mu_A(x)$ steeds noteren als $A(x)$. We zullen dus 
met andere woorden geen onderscheid meer maken tussen de vaagverzameling en de 
lidmaatschapsfunctie. Voor de klasse van vaagverzamelingen in een universum $X$ gebruiken we
de notatie $\mathcal{F}(X)$.

De \emph{drager} en de \emph{kern} van een vaagverzameling zijn twee belangrijke begrippen: 
\begin{definitie}
De drager van een vaagverzameling $A$ in $X$ wordt gedefinieerd als:
$$
supp\ A = \{x \in X \mid A(x) > 0\} 
$$
\end{definitie}
\begin{definitie}
De kern van een vaagverzameling $A$ in $X$ defini\"eren we als volgt:
$$
ker\ A = \{x \in X \mid A(x) = 1\}
$$
\end{definitie}
\noindent
Ook het begrip \emph{cardinaliteit} speelt vaak een belangrijke rol. De cardinaliteit van een 
een eindige scherpe verzameling wordt gegeven door het aantal elementen in die verzameling. 
Dit concept kan uitgebreid worden naar vaagverzamelingen door gebruik te maken van het begrip 
\emph{sigma count}:
\begin{definitie}
De sigma count van een vaagverzameling $A$ met eindige drager in een universum $X$ wordt
gedefinieerd door:
$$
|A|=\sum_{x \in X} A(x)
$$
\end{definitie}

\subsection{Bewerkingen op vaagverzamelingen}
\label{sectie:bew_op_vaagverz}

We beginnen met het defini\"eren van de begrippen \emph{negator}, \emph{conjunctor} en 
\emph{disjunctor}. Deze operatoren zijn uitbreidingen van de klassieke logische operatoren
$\lnot$ (negatie), $\land$ (conjunctie) en $\lor$ (disjunctie).
\begin{definitie}
Een negator $\mathcal{N}$ op $[0,1]$ is een dalende $[0,1] - [0,1]$ afbeelding die voldoet
aan de randvoorwaarden $\mathcal{N}(0)=1$ en $\mathcal{N}(1)=0$. 
\end{definitie}
\begin{definitie}
Een conjunctor $\mathcal{C}$ op $[0,1]$ is een stijgende $[0,1]^2 - [0,1]$ afbeelding die voldoet aan de
randvoorwaarden $\mathcal{C}(0,0)=\mathcal{C}(0,1)=\mathcal{C}(1,0)=0$ en $\mathcal{C}(1,1)=1$. 
Als een conjunctor voldoet aan 
$(\forall a \in [0,1])(\mathcal{C}(1,a)=\mathcal{C}(a,1)=a)$ dan spreekt men van een semi-norm.
Een driehoeksnorm (t-norm) is een commutatieve en associatieve semi-norm.
\end{definitie}
\begin{definitie}
Een disjunctor $\mathcal{D}$ op $[0,1]$ is een stijgende $[0,1]^2 - [0,1]$ afbeelding die voldoet
aan de randvoorwaarden $\mathcal{D}(1,0)=\mathcal{D}(0,1)=\mathcal{D}(1,1)=1$ en 
$\mathcal{D}(0,0)=0$. Als een disjunctor voldoet aan 
$(\forall a \in [0,1])(\mathcal{D}(0,a)=\mathcal{D}(a,0)=a)$ dan spreekt met van een semi-conorm.
Een driehoeksconorm (t-conorm) is een commutatieve en associatieve semi-conorm.
\end{definitie}

De meest gebruikte negator is de standaardnegator $N_s$. Het minimum $T_M$ en het algebra\"isch 
product $T_P$ zijn veelgebruikte t-normen. Bij de t-conormen zijn het maximum $S_M$ en de
probabilistische som $S_P$ dan weer populaire mogelijkheden. Deze operatoren worden als 
volgt gedefinieerd:
$$
\begin{array}{r@{\quad=\quad}l}
N_s(x) & 1 - x \\
T_M(x,y) & min(x,y) \\
T_P(x,y) & x \cdot y \\
S_M(x,y) & max(x,y) \\
S_P(x,y) & x +y - x \cdot y,
\end{array}
$$
voor alle $(x,y)$ in $[0,1]^2$.

We kunnen de bovenstaande operatoren nu gebruiken om de klassieke verzameltechnische bewerkingen 
$co$ (complement), $\cap$ (doorsnede) en $\cup$ (unie) te
veralgemenen tot bewerkingen op vaagverzamelingen.
\begin{definitie}
Het $\mathcal{N}$-complement $co_\mathcal{N} A$ van een vaagverzameling $A$ in $X$ wordt gedefinieerd
door de volgende vaagverzameling in X:
$$
(co_\mathcal{N} A)(x) = \mathcal{N}(A(x)),
$$
voor alle $x$ in $X$, met $\mathcal{N}$ een negator.
\end{definitie}
\begin{definitie}
De $\mathcal{C}$-doorsnede $A \cap_\mathcal{C} B$ van twee vaagverzamelingen $A$ en $B$ in $X$
wordt gedefinieerd door de volgende vaagverzameling in $X$:
$$
(A \cap_\mathcal{C} B)(x) = \mathcal{C}(A(x),B(x)),
$$
voor alle $x$ in $X$, met $\mathcal{C}$ een conjunctor.
\end{definitie}
\begin{definitie}
De $\mathcal{D}$-unie $A \cup_\mathcal{D} B$ van twee vaagverzamelingen $A$ en $B$ in $X$ wordt gedefinieerd door de
volgende vaagverzameling in $X$:
$$
(A \cup_\mathcal{D} B)(x) = \mathcal{D}(A(x),B(x)),
$$
voor alle $x$ in $X$, met $\mathcal{D}$ een disjunctor.
\end{definitie}

In het vervolg van deze scriptie zullen we meestal 
$(\mathcal{N},\mathcal{C},\mathcal{D})=(N_s,T_M,S_M)$ kiezen. We voeren daarom de volgende
verkorte notaties in:
$$
\begin{array}{r@{\quad=\quad}l}
A^c 		& co_{N_s} A \\
A \cap B 	& A \cap_{T_M} B \\
A \cup B	& A \cup_{S_M} B,
\end{array}
$$
waarbij $A$ en $B$ vaagverzamelingen zijn.


\subsection{L-vaagverzamelingen}

Het begrip vaagverzameling kan op zijn beurt nog eens uitgebreid worden tot 
\emph{L-vaagverzameling}. Deze uitbreiding is gebaseerd op het concept \emph{tralie}. We geven
hieronder eerst de definitie van dit concept, samen met de definities van de aanverwante
begrippen \emph{poset}, \emph{supremum} en \emph{infimum}. Daarna kunnen geven we dan, gebruik
maken van dit voorbereid werk, de definitie van een L-vaagverzameling. 
\begin{definitie}
Een partieel geordende verzameling (poset) is een koppel $(P,\le)$ bestaande uit een niet-ledige
verzameling $P$ en een binaire relatie $\le$ over $P$ die reflexief, antisymmetrisch en 
transitief is:
\begin{itemize}
  \item[(P.1)] $(\forall x \in P)(x \le x)$
  \item[(P.2)] $(\forall (x,y) \in P^2)(x \le y \land y \le x \Rightarrow x = y)$
  \item[(P.3)] $(\forall (x,y,z) \in P^3)(x \le y \land y \le z \Rightarrow x \le z)$
\end{itemize}
De relatie $\le$ noemt men een (parti\"ele) orderelatie over $P$.
\end{definitie} 
\begin{definitie}
Zij $(P,\le)$ een poset en $A \subseteq P$. Een element $b$ van $P$ is een bovengrens voor $A$
als en slechts als $(\forall a \in A)(a \le b)$. Als bovendien geldt $b \in A$, dan noemen
we $b$ het grootste element van $A$. De kleinste bovengrens voor $A$ is het
supremum voor $A$.
\end{definitie}
\begin{definitie}
Zij $(P,\le)$ een poset en $A \subseteq P$. Een element $b$ van $P$ is een ondergrens voor $A$
als en slechts als $(\forall a \in A)(b \le a)$. Als bovendien geldt $b \in A$, dan noemen
we $b$ het kleinste element van $A$. De grootste ondergrens voor $A$ is het
infimum voor $A$.
\end{definitie}
\begin{definitie}
Een poset $(L,\le)$ waarin elk doubleton een supremum en een infimum bezit, noemt men een tralie 
(lattice). Als elke niet-ledige deelverzameling van $L$
een supremum en een infimum bezit, dan spreekt men van een complete tralie.
\end{definitie}
% ALTERNATIEVE DEFINITIE VAN TRALIE
%% enkele nieuwe commando's die we verder in deze sectie nodig zullen hebben
%\newcommand{\dotand}{\ensuremath{\mathaccent\ldotp\land}}
%\newcommand{\dotor}{\ensuremath{\mathaccent\cdotp\lor}}
%\begin{definitie}
%Een algebra\"ische structuur $(L,\dotor,\dotand)$ bestaande uit een niet-ledige
%verzameling $L$ en twee binaire bewerkingen op $L$ wordt een tralie genoemd als en slechts als
%voor elke $a$, $b$ en $c$ uit $L$ geldt:
%\begin{itemize}
%  \item[(L.1)] $a \dotand a = a$ en $a \dotor a = a$
%  \item[(L.2)] $a \dotand b = b \dotand a$ en $a \dotor b = b \dotor a$
%  \item[(L.3)] $a \dotand (b \dotand c) = (a \dotand b) \dotand c$ en $a \dotor (b \dotor c) = (a \dotor b) \dotor c$
%  \item[(L.4)] $a \dotand (a \dotor b) = a$ en $a \dotor (a \dotand b) = a$
%\end{itemize}
%\end{definitie}

\begin{definitie}
Zij $X$ een universum en $(L,\le)$ een complete tralie. Een L-vaagverzameling $A$ in $X$ wordt
gekarakteriseerd door een $X - L$ afbeelding $\mu_A$:
$$
\begin{array}{lllll}
\mu_A: 	& X & \to 		& L	& \\
		& x & \mapsto 	& \mu_A(x),		& \forall x \in A
\end{array}
$$
\end{definitie}
\noindent
De klasse van scherpe verzamelingen in een universum $X$ stellen we voor door 
$\mathcal{L}(X)$. Merk op dat deze klasse zich herleidt tot $\mathcal{F}(X)$ voor $L = [0,1]$.

In deze scriptie zullen we enkel gebruik maken van L-vaagverzamelingen waarbij
$L = [0,1]^n$ met $n > 0$. Voor deze speciale gevallen kunnen we het begrip sigma count als volgt 
uitbreiden:
$$
|A|=\frac{1}{\sqrt{n}}\sum_{x \in X}\sqrt{(A_1(x))^2+(A_2(x))^2+\ldots+(A_n(x))^2},
$$
met $A(x)=(A_1(x),A_2(x),\ldots,A_n(x))$, $\forall x \in X$.  

\subsection{Bewerkingen op L-vaagverzamelingen}

In deze sectie gaan we de begrippen \emph{negator}, \emph{conjunctor} en 
\emph{disjunctor} uitbreiden naar L-vaag\-ver\-za\-me\-ling\-en. Deze uitbreiding heeft als gevolg dat de
de veralgemeende verzameltechnische bewerkingen $co$, $\cap$ en 
$\cup$ uit \ref{sectie:bew_op_vaagverz} eveneens toepasbaar zijn op L-vaagverzamelingen.

Zij $(L,\le)$ een complete tralie die $l$ als kleinste en $u$ als grootste element heeft. We
defini\"eren:
\begin{definitie}
Een negator $\mathcal{N}$ op $L$ is een dalende $L - L$ afbeelding die voldoet
aan de randvoorwaarden $\mathcal{N}(l)=u$ en $\mathcal{N}(u)=l$. 
\end{definitie}
\begin{definitie}
Een conjunctor $\mathcal{C}$ op $L$ is een stijgende $L^2 - L$ afbeelding die voldoet aan de
randvoorwaarden $\mathcal{C}(l,l)=\mathcal{C}(l,u)=\mathcal{C}(u,l)=l$ en $\mathcal{C}(u,u)=u$. 
Als een conjunctor voldoet aan 
$(\forall a \in L)(\mathcal{C}(u,a)=\mathcal{C}(a,u)=a)$ dan spreekt men van een semi-norm.
Een driehoeksnorm (t-norm) is een commutatieve en associatieve semi-norm.
\end{definitie}
\begin{definitie}
Een disjunctor $\mathcal{D}$ op $L$ is een stijgende $L^2 - L$ afbeelding die voldoet
aan de randvoorwaarden $\mathcal{D}(u,l)=\mathcal{D}(l,u)=\mathcal{D}(u,u)=u$ en 
$\mathcal{D}(l,l)=l$. Als een disjunctor voldoet aan 
$(\forall a \in L)(\mathcal{D}(l,a)=\mathcal{D}(a,l)=a)$ dan spreekt met van een semi-conorm.
Een driehoeksconorm (t-conorm) is een commutatieve en associatieve semi-conorm.
\end{definitie}

Merk op dat we in de bijzondere gevallen waarbij $L=[0,1]^n$ met $n > 0$, 
$(\mathcal{N},\mathcal{C},\mathcal{D})=(N_s,T_M,S_M)$ kunnen kiezen.


\section{Similariteitsmaten}

\section{Aggregatieoperatoren} 