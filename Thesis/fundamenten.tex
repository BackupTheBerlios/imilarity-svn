\chapter{Wiskundige fundamenten}

In dit hoofdstuk introduceren we eerst enkele basisbegrippen uit de vaagverzamelingenleer. 
Daarna geven we een overzicht van de similariteitsmaten en aggregatieoperatoren waarvan
we in het vervolg van deze scriptie gebruik zullen maken.

\section{Vaagverzamelingen}

De collectie van alle mogelijke elementen noemen we het \emph{universum} (bijvoorbeeld de
natuurlijke getallen). Een verzameling bevat bepaalde elementen uit dit universum (bijvoorbeeld de 
verzameling van de priemgetallen). 

In het geval van een \emph{scherpe verzameling}, behoort elk 
element uit het universum wel of niet tot de verzameling. Andere mogelijkheden zijn er
niet. Een dergelijke verzameling kan bijgevolg 
gerepresenteerd worden door een \emph{karakteristieke afbeelding}, die elk element uit het 
universum afbeeldt op 0 of 1. Dit getal noemen we de \emph{lidmaatschapsgraad} van het element 
in kwestie. De klasse van scherpe verzamelingen in een universum $X$ stellen we voor door 
$\mathcal{P}(X)$.
\begin{definitie}
Zij $X$ een universum. De karakteristieke afbeelding $\mu_A$ van een scherpe verzameling $A$ in $X$
wordt gedefinieerd als de $X - \{0,1\}$ afbeelding:
$$
\begin{array}{lllll}
\mu_A: 	& X & \to 		& \{0,1\}	& \\
		& x & \mapsto 	& 1,		& \textrm{ als } x \in A \\
		& x & \mapsto 	& 0,		& \textrm{ als } x \notin A
\end{array}
$$
\end{definitie}

Bij een \emph{vaagverzameling} kunnen alle waarden tussen 0 en 1 als lidmaatschapsgraad 
voorkomen. De karakteristiek afbeelding is in dit geval dus een $X - [0,1]$ afbeelding:
\begin{definitie}
Zij $X$ een universum. Een vaagverzameling $A$ in $X$ wordt gekarakteriseerd door een $X - [0,1]$
afbeelding  $\mu_A$:
$$
\begin{array}{lllll}
\mu_A: 	& X & \to 		& [0,1]	& \\
		& x & \mapsto 	& \mu_A(x),		& \forall x \in A
\end{array}
$$
\end{definitie}
\noindent
Een element $x \in X$ behoort dus tot de vaagverzameling $A$ met lidmaatschapsgraad $\mu_A(x)$.
Voor de eenvoud zullen we in het vervolg $\mu_A(x)$ steeds noteren als $A(x)$. We zullen dus 
met andere woorden geen onderscheid meer maken tussen de vaagverzameling en de 
lidmaatschapsfunctie. Voor de klasse van vaagverzamelingen in een universum $X$ gebruiken we
de notatie $\mathcal{F}(X)$.

De \emph{drager} en de \emph{kern} van een vaagverzameling zijn twee belangrijke begrippen: 
\begin{definitie}
De drager van een vaagverzameling $A$ in $X$ wordt gedefinieerd als:
$$
supp\ A = \{x \in X \mid A(x) > 0\} 
$$
\end{definitie}
\begin{definitie}
De kern van een vaagverzameling $A$ in $X$ defini\"eren we als volgt:
$$
ker\ A = \{x \in X \mid A(x) = 1\}
$$
\end{definitie}
\noindent
Ook het begrip \emph{cardinaliteit} speelt vaak een belangrijke rol. De cardinaliteit van een 
een eindige scherpe verzameling wordt gegeven door het aantal elementen in die verzameling. 
Dit concept kan uitgebreid worden naar vaagverzamelingen door gebruik te maken van het begrip 
\emph{sigma count}:
\begin{definitie}
De sigma count van een vaagverzameling $A$ met eindige drager in een universum $X$ wordt
gedefinieerd door:
$$
|A|=\sum_{x \in X} A(x)
$$
\end{definitie}

\section{Bewerkingen op vaagverzamelingen}

We beginnen met het defini\"eren van de begrippen \emph{negator}, \emph{conjunctor} en 
\emph{disjunctor}. Deze operatoren zijn uitbreidingen van de klassieke logische operatoren
$\lnot$ (negatie), $\land$ (conjunctie) en $\lor$ (disjunctie).
\begin{definitie}
Een negator $\mathcal{N}$ op $[0,1]$ is een dalende $[0,1] - [0,1]$ afbeelding die voldoet
aan de randvoorwaarden $\mathcal{N}(0)=1$ en $\mathcal{N}(1)=0$. 
\end{definitie}
\begin{definitie}
Een conjunctor $\mathcal{C}$ op $[0,1]$ is een stijgende $[0,1]^2 - [0,1]$ afbeelding die voldoet aan de
randvoorwaarden $\mathcal{C}(0,0)=\mathcal{C}(0,1)=\mathcal{C}(1,0)=0$ en $\mathcal{C}(1,1)=1$.
\end{definitie}
\begin{definitie}
Een disjunctor $\mathcal{D}$ op $[0,1]$ is een stijgende $[0,1]^2 - [0,1]$ afbeelding die voldoet
aan de randvoorwaarden $\mathcal{D}(1,0)=\mathcal{D}(0,1)=\mathcal{D}(1,1)=1$ en 
$\mathcal{D}(0,0)=0$.
\end{definitie}

De meest gebruikte negator is de standaardnegator $N_S$. Het minimum $C_M$ en het algebra\"isch 
product $C_P$ zijn veelgebruikte conjunctors. Bij de disjunctors zijn het maximum $D_M$ en de
probabilistische som $D_P$ dan weer populaire mogelijkheden. Deze operatoren worden als 
volgt gedefinieerd:
$$
\begin{array}{r@{\quad=\quad}l}
N_S(x) & 1 - x \\
C_M(x,y) & min(x,y) \\
C_P(x,y) & x \cdot y \\
D_M(x,y) & max(x,y) \\
D_P(x,y) & x +y - x \cdot y,
\end{array}
$$
voor alle $(x,y)$ in $[0,1]^2$.

We kunnen de bovenstaande operatoren nu gebruiken om de klassieke verzameltechnische bewerkingen 
$co$ (complement), $\cap$ (doorsnede) en $\cup$ (unie) te
veralgemenen tot bewerkingen op vaagverzamelingen.
\begin{definitie}
Het $\mathcal{N}$-complement $co_\mathcal{N} A$ van een vaagverzameling $A$ in $X$ wordt gedefinieerd
door de volgende vaagverzameling in X:
$$
(co_\mathcal{N} A)(x) = \mathcal{N}(A(x)),
$$
voor alle $x$ in $X$, met $\mathcal{N}$ een negator.
\end{definitie}
\begin{definitie}
De $\mathcal{C}$-doorsnede $A \cap_\mathcal{C} B$ van twee vaagverzamelingen $A$ en $B$ in $X$
wordt gedefinieerd door de volgende vaagverzameling in $X$:
$$
(A \cap_\mathcal{C} B)(x) = \mathcal{C}(A(x),B(x)),
$$
voor alle $x$ in $X$, met $\mathcal{C}$ een conjunctor.
\end{definitie}
\begin{definitie}
De $\mathcal{D}$-unie $A \cup_\mathcal{D} B$ van twee vaagverzamelingen $A$ en $B$ in $X$ wordt gedefinieerd door de
volgende vaagverzameling in $X$:
$$
(A \cup_\mathcal{D} B)(x) = \mathcal{D}(A(x),B(x)),
$$
voor alle $x$ in $X$, met $\mathcal{D}$ een disjunctor.
\end{definitie}

In het vervolg van deze scriptie zullen we doorgaands 
$(\mathcal{N},\mathcal{C},\mathcal{D})=(N_S,C_M,D_M)$ kiezen. We voeren daarom de volgende
verkorte notaties in:
$$
\begin{array}{r@{\quad=\quad}l}
A^c 		& co_{N_S} A \\
A \cap B 	& A \cap_{C_M} B \\
A \cup B	& A \cup_{D_M} B,
\end{array}
$$
waarbij $A$ en $B$ vaagverzamelingen zijn.


\section{L-vaagverzamelingen}

\section{Aggregatieoperatoren} 